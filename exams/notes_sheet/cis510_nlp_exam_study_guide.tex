\documentclass[9pt]{extarticle}
\usepackage{comment} % enables the use of multi-line comments (\ifx \fi)
% \usepackage{fullpage} % changes the margin
\usepackage[margin=0.3in]{geometry}
\usepackage[utf8]{inputenc}
\usepackage{amsmath}
\usepackage{multicol}

\usepackage{subcaption}

\usepackage{titlesec}  % Used to adjust title heading
% Format:  \titlespacing{command}{left spacing}{before spacing}{after spacing}[right]
\titlespacing\section{0pt}{6pt plus 4pt minus 0pt}{4pt plus 2pt minus 0pt}

\setlength\parindent{0pt}
\setlength\itemsep{0pt}

\newcommand{\horizontalbreak}{
  {
    \begin{center}
      \noindent\rule{7.5in}{0.4pt}
    \end{center}
  }
}

\newcommand{\etal}{et~al.}
\newcommand{\elkan}{Elkan \&~Noto}

% Used for including standalone docs
\usepackage{standalone}

\newcommand{\transpose}{^{\text{T}}}

% Imported via UltiSnips
% Unbreakable dash:
%  Words hyphened with these dashes can also be broken at other positions than the dash
%    \-/ hyphen
%    \-- en-dash
%    \--- em-dash
%    extdash unbreakable dashes
%
%  No line breaks possible at the hyphen
%    \=/ hyphen
%    \== en-dash
%    \=== em-dash
\usepackage[shortcuts]{extdash}

% Imported via UltiSnips
\usepackage[dvipsnames]{xcolor}
\newcommand{\colortext}[2]{{\color{#1} #2}}
\newcommand{\red}[1]{\colortext{red}{#1}}
\newcommand{\blue}[1]{\colortext{blue}{#1}}
\newcommand{\green}[1]{\colortext{green}{#1}}

% Imported via UltiSnips
\usepackage{amsmath}
\DeclareMathOperator*{\argmax}{arg\,max}
\DeclareMathOperator*{\argmin}{arg\,min}
\usepackage{amsfonts}  % Used for \mathbb and \mathcal
\usepackage{amssymb}

% Imported via UltiSnips
\usepackage{mathtools} % for "\DeclarePairedDelimiter" macro
% \swapifbranches changes unstarred paired delimiters to starred and
% vice versa.  This means by default, paired delimiters have the star.
\usepackage{etoolbox}
\newcommand\swapifbranches[3]{#1{#3}{#2}}
\makeatletter
\MHInternalSyntaxOn
\patchcmd{\DeclarePairedDelimiter}{\@ifstar}{\swapifbranches\@ifstar}{}{}
\MHInternalSyntaxOff
\makeatother
% Place after swap to ensure swap star
\DeclarePairedDelimiter{\sbrack}{\lbrack}{\rbrack}
\DeclarePairedDelimiter{\floor}{\lfloor}{\rfloor}
\DeclarePairedDelimiter{\ceil}{\lceil}{\rceil}
\DeclarePairedDelimiter{\abs}{\lvert}{\rvert}
\DeclarePairedDelimiter{\norm}{\lVert}{\rVert}
\usepackage{bm}
\DeclarePairedDelimiterX\set[1]\lbrace\rbrace{#1}
\DeclarePairedDelimiterX\setbuild[2]\lbrace\rbrace{#1 \bm: #2}
\newcommand{\ints}[1]{{\sbrack{#1}}}
\newcommand{\func}[3]{{#1:#2\rightarrow#3}}
\newcommand{\defeq}{\stackrel{\mathclap{\mbox{\tiny def}}}{=}}

% Imported via UltiSnips
\usepackage{multirow}
\usepackage{booktabs}

% Imported via UltiSnips
\usepackage{tikz}
\usetikzlibrary{arrows,decorations.markings,shadows,positioning,calc,backgrounds,shapes}

% Imported via UltiSnips
\usepackage[noend]{algpseudocode}
\usepackage[Algorithm,ruled]{algorithm}
\algnewcommand\algorithmicforeach{\textbf{for each}}
\algdef{S}[FOR]{ForEach}[1]{\algorithmicforeach\ #1\ \algorithmicdo}
\newcommand{\algin}[1]{\hspace*{\algorithmicindent} \textbf{Input} #1\\}
\newcommand{\algout}[1]{\hspace*{\algorithmicindent} \textbf{Output} #1\\}

\usepackage[dvipsnames]{xcolor}
\renewcommand{\green}[1]{{\color{ForestGreen} #1}}

\begin{document}
\begin{center}
CIS510 NLP Exam Notes Sheet -- Zayd Hammoudeh
\end{center}
\begin{multicols}{3}
  \section*{Lecture~\#1: Introduction}

  \textbf{Challenges of NLP}:
  \begin{itemize}
    \item Language is \textit{discrete}. Many CV techniques do not work
    \item Language is \textit{compositonal}. Meaning comes form understanding individual words \& how they combine
    \item \textit{Ambiguity}:
  \end{itemize}

  \section*{Lecture~\#2: Text Classification}
  \textit{Examples}: Topic, sentiment, language, authorship

  \begin{equation*}\label{eq:L02:NaiveBayes}
    \argmax_c \Pr\sbrack{c \vert X} = \argmax_{c} \frac{\Pr[X \vert c] \Pr[c]}{\Pr[X]}
  \end{equation*}
  \begin{equation*}\label{eq:L02:Bernoulli}
    \Pr[c] = \frac{\text{\#docs label } c \text{ in } D}{\text{\#docs in } D}
  \end{equation*}

  \section*{Lecture~\#3: Word Embeddings}
  \textit{Distributional Semantics}: A word's meaning is given by words frequently appear close-by.

  \textit{Context}: Set of words that appear nearby in fixed-size window

  \textbf{\blue{Continuous Bag of Words}} (CBOW) ${\Pr\sbrack{w_t \vert w_{t-2}, w_{t-1}, w_{t+1}, w_{t+2}}}$

  \textbf{\blue{Skip-Grams}} (SG): \\ ${\Pr\sbrack{w_{t+i} \vert w_{t}}, i \in \set{-2,-1,1,2}}$

  \section*{Lecture~\#4: Deep Learning}
  \textbf{Logistic Regression}:
  \begin{equation*}\label{eq:L04:LogReg}
    \mathcal{L}(\mathbf{w},b) = -\sum_{\mathbf{x}_i,y_i \in \mathbf{d}} \log \sigma\big( y_i(\mathbf{w}\cdot \mathbf{x}_i + b) \big)
  \end{equation*}
  \section*{Lecture~\#5: Sequential Labeling}
  \subsection*{\blue{Hidden Markov Model}}

  \begin{align*}
    \argmax_{\vec{y}} \Pr\sbrack{\vec{y} \vert \vec{x}} &= \argmax_{\vec{y}} \frac{\Pr\sbrack{\vec{x} \vert \vec{y}}\Pr[\vec{y}]}{\Pr[\vec{x}]} \\
                                                        &= \argmax_{\vec{y}} \Pr\sbrack{\vec{x} \vert \vec{y}} \Pr[\vec{y}]
  \end{align*}

  \textbf{\green{First-Order Markov Assumption}}:

  \begin{equation*}
    \Pr\sbrack{y_{1},\ldots,y_{n}} = \prod_{i=1}^{n} \Pr[y_{i} \vert y_{i-1}]
  \end{equation*}

  \textbf{\green{Indepen.\ Assum}} $\Pr\sbrack{\vec{x} \vert \vec{y}} = \prod_{i=1}^{n} \Pr[x_{i} \vert y_{i}]$

  \textbf{\green{Transition Prob}}: $\Pr\sbrack{y_{t} \vert y_{t-1}}$

  \textbf{\green{Emission Prob}}: $\Pr\sbrack{x_{t} \vert y_{t}}$

  \subsection*{Max.\ Entropy Markov Model}
  \subsection*{CRF}
  \section*{Lecture~\#6: Constituent Parsing}
  \section*{Lecture~\#7: Dependency Parsing}

\end{multicols}


\end{document}
