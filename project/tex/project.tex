\documentclass{article}

% if you need to pass options to natbib, use, e.g.:
%     \PassOptionsToPackage{numbers, compress}{natbib}
% before loading neurips_2019

% ready for submission
% \usepackage{neurips_2019}

\newcommand{\etal}{et~al.}
\newcommand{\elkan}{Elkan \&~Noto}

% Used for including standalone docs
\usepackage{standalone}

\newcommand{\transpose}{^{\text{T}}}

% Imported via UltiSnips
% Unbreakable dash:
%  Words hyphened with these dashes can also be broken at other positions than the dash
%    \-/ hyphen
%    \-- en-dash
%    \--- em-dash
%    extdash unbreakable dashes
%
%  No line breaks possible at the hyphen
%    \=/ hyphen
%    \== en-dash
%    \=== em-dash
\usepackage[shortcuts]{extdash}

% Imported via UltiSnips
\usepackage[dvipsnames]{xcolor}
\newcommand{\colortext}[2]{{\color{#1} #2}}
\newcommand{\red}[1]{\colortext{red}{#1}}
\newcommand{\blue}[1]{\colortext{blue}{#1}}
\newcommand{\green}[1]{\colortext{green}{#1}}

% Imported via UltiSnips
\usepackage{amsmath}
\DeclareMathOperator*{\argmax}{arg\,max}
\DeclareMathOperator*{\argmin}{arg\,min}
\usepackage{amsfonts}  % Used for \mathbb and \mathcal
\usepackage{amssymb}

% Imported via UltiSnips
\usepackage{mathtools} % for "\DeclarePairedDelimiter" macro
% \swapifbranches changes unstarred paired delimiters to starred and
% vice versa.  This means by default, paired delimiters have the star.
\usepackage{etoolbox}
\newcommand\swapifbranches[3]{#1{#3}{#2}}
\makeatletter
\MHInternalSyntaxOn
\patchcmd{\DeclarePairedDelimiter}{\@ifstar}{\swapifbranches\@ifstar}{}{}
\MHInternalSyntaxOff
\makeatother
% Place after swap to ensure swap star
\DeclarePairedDelimiter{\sbrack}{\lbrack}{\rbrack}
\DeclarePairedDelimiter{\floor}{\lfloor}{\rfloor}
\DeclarePairedDelimiter{\ceil}{\lceil}{\rceil}
\DeclarePairedDelimiter{\abs}{\lvert}{\rvert}
\DeclarePairedDelimiter{\norm}{\lVert}{\rVert}
\usepackage{bm}
\DeclarePairedDelimiterX\set[1]\lbrace\rbrace{#1}
\DeclarePairedDelimiterX\setbuild[2]\lbrace\rbrace{#1 \bm: #2}
\newcommand{\ints}[1]{{\sbrack{#1}}}
\newcommand{\func}[3]{{#1:#2\rightarrow#3}}
\newcommand{\defeq}{\stackrel{\mathclap{\mbox{\tiny def}}}{=}}

% Imported via UltiSnips
\usepackage{multirow}
\usepackage{booktabs}

% Imported via UltiSnips
\usepackage{tikz}
\usetikzlibrary{arrows,decorations.markings,shadows,positioning,calc,backgrounds,shapes}

% Imported via UltiSnips
\usepackage[noend]{algpseudocode}
\usepackage[Algorithm,ruled]{algorithm}
\algnewcommand\algorithmicforeach{\textbf{for each}}
\algdef{S}[FOR]{ForEach}[1]{\algorithmicforeach\ #1\ \algorithmicdo}
\newcommand{\algin}[1]{\hspace*{\algorithmicindent} \textbf{Input} #1\\}
\newcommand{\algout}[1]{\hspace*{\algorithmicindent} \textbf{Output} #1\\}


% to compile a preprint version, e.g., for submission to arXiv, add add the
% [preprint] option:
\usepackage[final,nonatbib]{neurips_2019}

% to compile a camera-ready version, add the [final] option, e.g.:
%     \usepackage[final]{neurips_2019}

% to avoid loading the natbib package, add option nonatbib:
%     \usepackage[nonatbib]{neurips_2019}

\newcommand{\toolname}{ddPU}

\newcommand{\Repel}{Repellent}
\newcommand{\repel}{\MakeLowercase{\Repel}}

\newcommand{\X}{x}
\newcommand{\domain}{\real^{d}}
\newcommand{\eX}[1]{\X_{(#1)}}
\newcommand{\xA}{\eX{A}}
\newcommand{\xP}{\eX{P}}
\newcommand{\xN}{\eX{N}}

\newcommand{\y}{y}
\newcommand{\labels}{\set{{\pm}1}}
\newcommand{\pCls}{{+}1}
\newcommand{\nCls}{{-}1}

\newcommand{\latent}{s}
\newcommand{\params}{\theta}

\newcommand{\pDist}{p}
\newcommand{\joint}{\pDist_{XY}}
\newcommand{\marginal}{\pDist_{X}}

\newcommand{\size}{n}
\newcommand{\train}{\mathcal{X}}
\newcommand{\trainVar}[1]{\train_{\textnormal{#1}}}
\newcommand{\unlabel}{\trainVar{U}}
\newcommand{\pos}{\trainVar{P}}
% Batch specific naming
\newcommand{\cnt}{i}
\newcommand{\unlI}[1]{#1{\mathcal{X}}_{\textnormal{U}}^{(\cnt)}}
\newcommand{\posI}[1]{#1{\mathcal{X}}_{\textnormal{P}}^{(\cnt)}}


% Siamese Network Related Macros
\newcommand{\margin}{\alpha}
\newcommand{\SiamDim}{m}
\newcommand{\SiamFunc}{f}
\newcommand{\distMetric}{\delta}
\newcommand{\siamDist}[3]{\norm{#1\left(#2\right) - #1\left(#3\right)}}
\newcommand{\lTrip}{\mathcal{L}_{\text{Triplet}}}

% Function names for the DeepPU architecture
\newcommand{\fPU}{g}
\newcommand{\fPUenc}{\fPU_{\text{enc}}}
\newcommand{\fPUp}{\fPU_{p}}
\newcommand{\fPUn}{\fPU_{n}}


\usepackage[utf8]{inputenc} % allow utf-8 input
\usepackage[T1]{fontenc}    % use 8-bit T1 fonts
\usepackage{url}            % simple URL typesetting
\usepackage{amsfonts}       % blackboard math symbols
\usepackage{nicefrac}       % compact symbols for 1/2, etc.
\usepackage{microtype}      % microtypography

\usepackage{hyperref}       % hyperlinks

\title{Addressing Negative Covariate Shift \\ on 20~Newsgroups Classification}

% The \author macro works with any number of authors. There are two commands
% used to separate the names and addresses of multiple authors: \And and \AND.
%
% Using \And between authors leaves it to LaTeX to determine where to break the
% lines. Using \AND forces a line break at that point. So, if LaTeX puts 3 of 4
% authors names on the first line, and the last on the second line, try using
% \AND instead of \And before the third author name.

\author{%
  Zayd Hammoudeh \\
  Department of Computer \& Information Science \\
  University of Oregon \\
  Eugene, OR 97403 \\
  \texttt{\href{mailto:zayd@cs.uoregon.edu}{zayd@cs.uoregon.edu}}
}

\begin{document}

\maketitle

\begin{abstract}
  Collecting representative, labeled training data may be challenging in some text classification domains. This project investigates techniques to address negative-class covariate shift.  We compare the performance of text classifiers trained using the non-negative positive-unlabeled (nnPU) and positive, unlabeled, biased-negative (PUbN) risk estimators against standard supervised (positive-negative --~PN) baselines. Our results indicate that the marginal distribution bias type may determine which risk estimator performs best.
\end{abstract}

\documentclass[]{subfiles}

\begin{document}
\section{Introduction}\label{sec:Introduction}

Consider binary classification of text documents.  Each document is represented by two random variables: independent feature vector~${\X \in \domainX}$ and dependent label~${\y \in \domainY}$. The document population is generated from an unknown, joint probability distribution~${\joint(\X,\y)}$.  By Bayes' rule, the joint distribution decomposes as

\begin{align}
    \joint(\X,\y) &= \marginal(\X) \posterior \label{eq:TrainDist} \\
                  &= \pDist(\y = \pcls) \pDist(\X \vert \y = \pcls) + \pDist(\y = \ncls) \pDist(\X \vert \y = \ncls) \nonumber\\
                  &\equiv \prior \pcond(\X) + (1 - \prior) \ncond(\X) \label{eq:Joint:Bayes}
\end{align}

\noindent
where $\prior$~is the positive class prior probability while $\pcond$~and $\ncond$~are the positive and negative class-conditional distributions respectively.

Supervised binary classification's training set,~$\train$, traditionally consists of $\size$~independent samples from Eq.~\eqref{eq:Joint:Bayes}.  Therefore, the training set partitions as ${\train = \ptrain \sqcup \ntrain}$ where ${\ptrain \sim \pcond}$ are the positive-valued examples and ${\ntrain \sim \ncond}$ are negative-valued.  For that reason, this training paradigm is often referred to as \textit{positive-negative}~(PN) learning.

This idealized supervised training model often does not apply in practice.  Training set labeling can be expensive or difficult meaning there are few labeled examples but numerous unlabeled ones.  Additionally, the labeled set may not be representative of~$\joint$.  For example, the training examples only characterizes a small subset of $\joint$'s support, including where there is no labeled data at all for one class.

The \textit{empirical risk minimization} training framework assumes that a low training set expected risk correlates to low inference error. Training set bias, such as those aforementioned, undermine that assumption, and can lead to large, unpredictable test set error rates.

One of the most common types of training set bias is \textit{covariate shift}. Define $\joint$~\eqref{eq:TrainDist} and ${\joint'}$~\eqref{eq:TestDist} as the training and set joint distributions respectively.  Posterior distribution,~${\posterior}$ is identical in both cases; rather, they only differ in their marginal distributions, i.e.,~$\marginal$ and~${\marginal'}$.~\cite{Huang:2006}  This paper assumes covariate shift for only one class, i.e.,~the negative one.

\begin{equation}\label{eq:TestDist}
    \joint'(\X,\y) = \marginal'(\X) \posterior
\end{equation}

Previous work has taken different routes to overcome negative covariate shift's effects on text classification.  Li\etal~\cite{Li:2010} ignored the biased negative training data entirely and attempted to learn a classifier using only the positive and unlabeled sets -- an approach known as \textit{positive\-/unlabeled}~(PU) learning.  Fei \&~Liu~\cite{Fei:2015} took the alternate approach of ignoring any unlabeled data and learned a traditional supervised classifier using just the positive and biased-negative training examples.  Hsieh\etal~\cite{Hsieh:2018} jointly used the \underline{p}ositive, \underline{u}nlabeled, and \underline{b}iased \underline{n}egative~(PUbN) in a risk estimator that works with standard empirical risk minimization.

The primary contribution of this work is an empirical comparison of PN, PU, and~PUbN classifiers on the 20~newsgroups dataset~\cite{20newsgroups} under negative covariate shift.  The remainder of this document is structured as follows.  Section~\ref{sec:RiskEstimators} describes the PN, PU, and~PUBN risk estimation functions. Section~\ref{sec:Architectures} describes the two neural network architectures used. Section~\ref{sec:20newsgroups} provides a brief overview of the 20~newsgroups dataset including how it is used in this paper. Section~\ref{sec:ExperimentalResults} describes our experimental setup, baselines, and results.  We provide brief concluding comments in Section~\ref{sec:Conclusions}.
\end{document}

\section{Risk estimation}\label{sec:RiskEstimators}

Let~$\func{\dec}{\domainX}{\real}$ be a decision function from feature vector~$\X$ to a real number, and let~$\func{\loss}{\domainX \times \domainY}{\real_{{\geq}0}}$ be the loss function.  A \textit{risk estimator},~$\risk$,\footnote{Although formally a function of~$\dec$ as shown in Eq.~\eqref{eq:RiskEstimator:Expectation}, the decision function is dropped for notational brevity going forward.} quantifies the $\dec$'s~expected loss; formally,

\begin{equation}\label{eq:RiskEstimator:Expectation}
  \risk(\dec) = \mathbb{E}_{(\X,\y) \sim \joint}\sbrack{\floss{\decX}{\y}}\text{.}
\end{equation}

Since joint distribution~$\joint$ is unknown, the true expected risk is unknown.  Rather, an empirical estimate of the expected risk,~$\emprisk$ is used in practice.  In the remainder of this section, we provide an overview of the PN, PU (specifically nnPU), and PUbN risk estimators as well as their empirical estimates\footnote{All empirical risk estimates described here can be used with both batch and stochastic gradient descent}.

\subsection{PN --- positive-negative}

PN~classification has access to both positive and negative labeled examples.  Therefore, the risk estimator is exactly specified by Eq.~\eqref{eq:RiskEstimator:Expectation}.  Estimating the PN~empirical risk is straightforward as shown in Eq.~\eqref{eq:EmpRisk:PN}; it is merely the mean loss for all examples in positive-negative training set~$\train$.  This formulation applies irrespective of covariate shift, if any.

\begin{equation}\label{eq:EmpRisk:PN}
  \emprisk = \frac{1}{\abs{\train}} \sum_{(\X,\y) \in \train} \floss{\decX}{\y}
\end{equation}

\subsection{nnPU --- non-negative positive-unlabeled}

Since positive\-/unlabeled~(PU) learning has no negative labeled examples, the traditional supervised learning cannot be used. By Bayes' Rule in Eq.~\eqref{eq:Joint:Bayes}, the expected risk can be decomposed into the risk associated with each label (positive and negative) as shown in Eq.~\eqref{eq:Risk:Bayes}.  Note that ${\varrisk{D}{\ypred}}$ represents the expected risk for samples drawn from distribution~${\pDist_{D}}$ (e.g.,~\underline{p}ositive and \underline{n}egative class-conditional distributions) and the predicted label in loss function~$\loss$ is~$\ypred$.

\begin{align}
  \risk &= \prior \mathbb{E}_{\X \sim \pcond}\sbrack{\floss{\decX}{\pcls}} + (1-\prior) \mathbb{E}_{\X \sim \ncond}\sbrack{\floss{\decX}{\ncls}} \nonumber \\
        &= \prior \prisk{P} + (1-\prior) \nrisk{N} \label{eq:Risk:Bayes}
\end{align}

Since the unlabeled set is drawn from marginal distribution,~$\marginal$, it is clear that:

\begin{align}
  \nrisk{U} &= \mathbb{E}_{\X \sim \marginal} \sbrack{\floss{\X}{\ncls}} \nonumber \\
            &= \prior \mathbb{E}_{\X \sim \pcond} \sbrack{\floss{\X}{\ncls}} + (1 - \prior) \mathbb{E}_{\X \sim \ncond} \sbrack{\floss{\X}{\ncls}}\nonumber \\
            &= \prior \nrisk{P} + (1 - \prior) \nrisk{N} \label{eq:Risk:Unlabeled}
\end{align}

\noindent
Reformatting the above and combining the above with Eq.~\eqref{eq:Risk:Bayes} yields the unbiased positive\-/unlabeled~(uPU) risk estimator in Eq.~\eqref{eq:Risk:uPU}.~\cite{duPlessis:2014}

\begin{equation}\label{eq:Risk:uPU}
  \risk = \prior \prisk{P} + \nrisk{U} - \prior \nrisk{P}
\end{equation}

\paragraph{Non\-/negativity} Given the definitions of~$\loss$ and~$\risk$, it is clear that $\nrisk{N}$~should never fall below zero.  uPU's surrogate,~${\nrisk{U} - \prior \nrisk{P}}$, often goes negative with highly expressive learner such as neural networks.  Kiryo\etal~\cite{Kiryo:2017} proposed the non\-/negative positive\-/unlabeled~(nnPU) risk estimator in Eq.~\eqref{eq:Risk:nnPU}; the primary difference versus uPU is the negative risk surrogate is explicitly forced to a positive value via the~$\max$.

\begin{equation}\label{eq:Risk:nnPU}
  \risk = \prior \prisk{P} + \max\left\{0, \nrisk{U} - \prior \nrisk{P} \right\}
\end{equation}

Whenever the negative risk surrogate is less than~0, nnPU uses a special gradient ${-\gamma \nabla \prior \nrisk{P} - \nrisk{U}}$ where~${\gamma \in (0,1]}$ is a hyperparameter to attenuate the learning rate.  Observe that the negative risk surrogate is deliberately negated; this is done to ``defit'' the learner so that it no longer under estimates the negative class' expected risk.

Although forcing the negative surrogate induces an estimation bias (i.e.,~its expected value does not equal the true expectation), nnPU often performs better in practice and satisfies ERM uniform convergence.

\paragraph{Empirical Estimation} Each term in nnPU can be empirically estimated from the training set components where:

\begin{equation}\label{eq:EmpRisk:Pos}
  \evrisk{P}{\ypred} = \frac{1}{\abs{\ptrain}} \sum_{\X \in \ptrain} \floss{\decX}{\ypred}
\end{equation}

\noindent
and for unlabeled set ${\utrain \sim \marginal}$,

\begin{equation}
  \evrisk{U}{-} = \frac{1}{\abs{\utrain}} \sum_{\X \in \utrain} \floss{\decX}{\ncls} \text{.}
\end{equation}

\noindent
In all experiments, $\prior$~is a hyperparameter.

\subsection{PUbN --- positive, unlabeled, biased-negative}

Let $\latent$~be a latent random variable representing whether the corresponding tuple is eligible for labeling.  Therefore, the full joint distribution~$\trijoint$ becomes trivariate as it also includes this latent r.v.  By definition, ${\pDist(\latent = \pcls \vert \X, \y = \pcls) = 1}$ (i.e.,~no positive bias) or equivalently ${\pDist(\y = \ncls \vert \X, \latent = \ncls) = 1}$.  The biased negative conditional distribution is therefore~${\bncond = \pDist(\X \vert \y = \ncls, \latent = \pcls)}$.

The marginal distribution can be partitioned as

\begin{equation*}
  \marginal = \pDist(\X, \y = \pcls) + \pDist(\X, \y = \ncls, \latent = \pcls) + \pDist(\X, \y = \ncls, \latent = \ncls) \text{.}
\end{equation*}

\noindent
The expected risk therefore becomes

\begin{equation}\label{eq:Risk:WithBN}
  \risk = \prior \prisk{P} + \plabel \nrisk{bN} + (1 - \prior - \rho) \smrisk
\end{equation}

\noindent
where ${\plabel = \pDist(\y = \ncls, \latent = \pcls)}$ is a hyperparameter.

Define ${\sigma(\X) = \pDist(\latent = \pcls \vert \X)}$.  While the proof is well beyond the scope of this document, Hsieh\etal~\cite{Hsieh:2018} demonstrated that with guaranteed estimation error bounds $\smrisk$~decomposes as

\begin{equation}\label{eq:ExpectedRisk:PUbN:Latent}
  \begin{aligned}
    \smrisk = &\mathbb{E}_{\X \in \marginal}\sbrack{\mathbbm{1}_{\sigX \leq \eta} \floss{\decX}{\ncls} \sigdiff} \\
              &+ \prior \mathbb{E}_{\X \sim \pcond} \sbrack{\mathbbm{1}_{\sigX > \eta} \floss{\decX}{\ncls} \frac{\sigdiff}{\sigX}} \\
              &+ \plabel \mathbb{E}_{\X \sim \pDist(\X \vert \latent = \pcls, \y = \ncls)} \sbrack{\mathbbm{1}_{\sigX > \eta} \floss{\decX}{\ncls} \frac{\sigdiff}{\sigX}}
  \end{aligned}
\end{equation}

\noindent
where $\mathbbm{1}$~is the indicator function and $\eta$~is a hyperparameter that controls the importance of unlabeled data versus $\textnormal{P}$ or~$\textnormal{bN}$ data.

\paragraph{Empirical Estimation} $\prisk{P}$~and~$\nrisk{bN}$ can be estimated directly from~$\ptrain$ and~$\bntrain$.  Estimating~$\smrisk$ is more challenging and actually requires the training of two classifiers.

$\sigX$~can be empirically estimated by training a probabilistic classifier of ${\ptrain \sqcup \bntrain}$ versus~$\utrain$ using nnPU; refer to this learned approximation as~$\hsigX$.  Probabilistic classifiers must be adequately calibrated to generate probabilities.  Hsieh\etal\ achieve this by training using the logistic loss during this stage of training.

As mentioned previously, $\eta$~is a hyperparameter.  Rather than specifying it directly, Hsieh\etal instead specify a hyperparameter~$\tau$ and calculate~$\eta$ from

\begin{equation}\label{eq:EtaCalculation}
\abs{\setbuild{\X \in \utrain}{\hsigX \leq \eta}} = \tau (1 - \prior - \plabel)\abs{\utrain} \text{.}
\end{equation}

\noindent
This approach provides a more intuitive insight into the balance between~$\utrain$ and $\ptrain$/$\bntrain$.

The expected risk for examples where ${\latent = \ncls}$ is empirically estimated via

\begin{equation}\label{eq:EmpRisk:PUbN:Latent}
  \begin{aligned}
    \smrisk = &\frac{1}{\abs{\utrain}} \sum_{\xvar{U} \in \utrain} \sbrack{\mathbbm{1}_{\hsig(\xvar{U}) \leq \eta} \floss{\dec(\xvar{U})}{\ncls} \big(1 - \hsig(\xvar{U})\big)} \\
              &+\frac{\prior}{\abs{\ptrain}} \sum_{\xvar{P} \in \ptrain} \sbrack{\mathbbm{1}_{\hsig(\xvar{P}) > \eta} \floss{\dec(\xvar{P})}{\ncls} \frac{1 - \hsig(\xvar{P})}{\hsig(\xvar{P})}} \\
              &+\frac{\plabel}{\abs{\bntrain}} \sum_{\xvar{bN} \in \bntrain} \sbrack{\mathbbm{1}_{\hsig(\xvar{bN}) > \eta} \floss{\dec(\xvar{bN})}{\ncls} \frac{1 - \hsig(\xvar{bN})}{\hsig(\xvar{bN})}} \text{.}
  \end{aligned}
\end{equation}

\section{Classifier neural architectures}\label{sec:Architectures}

Risk estimators are generally classifier agnostic.  They can be coupled with whatever learning paradigm imparts the most advantageous inductive bias. When labeled data is limited (including being non\-/existent for some classes), transfer learning often improves a classifier's performance by leveraging portable feature information learned from other (labeled) datasets.

This section proposes two 20~newsgroups neural classifier architectures.  Section~\ref{sec:Architectures:LSTM}'s LSTM model only makes limited use of transfer learning but is more flexible overall.  Section~\ref{sec:Architectures:ELMo}'s ELMo~model depends almost entirely on transfer learning and is more rigid. A comparison of each architecture's empirical results is provided in Section~\ref{sec:ExperimentalResults}.

\subsection{End-to-end LSTM network}\label{sec:Architectures:LSTM}

This architecture is the ``classic'' deep recurrent NLP classifier. Document tokens are encoded using an embedding (e.g,~GloVe), and these dense representations are input to a recurrent block composed of one or more bidirectional LSTM layers.  The final recurrent block output is passed to a feedforward network with one or more hidden layers and ReLU activation.

This model's only transferred knowledge is the (GloVe)~embedding matrix. All LSTM and feedforward parameters must be learned from scratch.  Given the limited labeled data, this may be onercous but as a positive it allows the learner significant flexibility.  We experimented with freezing and unfreezing the embedding weights, but saw no major performance difference with either approach.

\subsection{ELMo preprocessed vectors}\label{sec:Architectures:ELMo}

The traditional word embedding model treats each token (word) as context independent and yield token-specific representations. Proposed in~2018 by Peters\etal~\cite{Peters:2018}, ELMo (\underline{e}mbeddings for \underline{l}anguage \underline{mo}dels) extends the word embedding model such that each token representation is \textit{context dependent} --- a function of the current token as well as the entire preceding input.  ELMo uses unsupervised learning allowing it to be pretrained on a large scale.  ELMo representations easily integrate into many existing NLP~architectures, and often can be swapped for GloVe or other word embeddings.

The Allen Institute for AI provides multiple pretrained ELMo models.  We selected the best performing ELMo network --- the original architecture trained on the extended 5.5~billion token corpus, specifically 1.9~billion tokens from Wikipedia and 3.6~billion from the WMT monolingual news crawl. At over 93~million parameters, combining this ELMo model with long 20~newsgroups documents causes GPU memory overflow errors on the University of Oregon's K80~GPUs.  This limited our of use of ELMO to just creating a single static vector for each document --- in essence a ``document embedding.''

The original ELMo embedder model has three layers --- a character CNN (for improved unknown word robustness) followed by two bidirectional LSTMs.  The layer's output is a 1,024\-/dimension vector.  Given an input stream of $m$~tokens, the embedder's output is a tensor of size~${\langle \textnormal{\#Layers} \times d_{\textnormal{layer}} \times \textnormal{\#Tokens} \rangle}$ --- in this case~${\langle 3 \times 1024 \times m \rangle}$.

We used the R\"{u}ckl\'{e}\etal's~\cite{Ruckle:2018} paradigm where a sentence representation is formed by taking the minimum, maximum, and average value along each layer output dimension in a tensor like above. Therefore, our model turns each document into a vector of dimension ${\abs{\set{\max,\min,\textnormal{avg}}} \cdot \textnormal{\#Layers} \cdot d_{\textnormal{layer}} = 3 \cdot 3 \cdot 1024 = 9,216}$.

Each aforementioned static document representation is input into a feedforward network with two hidden layers of dimension~300 and ReLU~activations. This feedforward block contains the only volatile parameters making each epoch very fast~(<500ms in our experiments).\footnote{This ELMo\-/based architecture was first proposed by Hsieh\etal~\cite{Hsieh:2018}.}

\subsection{Implementation}

Irrespective of the classifier architecture, each newsgroup document was tokenized using~\texttt{nltk}. Both architectures are implemented in PyTorch with the source code available on~\href{\sourceCode}{Github}.  In our implementation, the LSTM~architecture is the default. To instead use preprocessed ELMo~vectors, set command line flag~\texttt{\mbox{\-/\-/preprocessed}}.

\section{20~newsgroups}\label{sec:20newsgroups}

First published by Lang in 1995, the ``20~newsgroups'' dataset is a collection of internet discussion board posts.  The original dataset consisted of 20,000~documents~\cite{20newsgroups} but was pruned to 18,828 documents in 2007 after duplicates and cross\-/posts were removed~\cite{Rennie:2007}.   The dataset's latest version has a predefined split of 11,314~train and 7,532~test documents.

Each document is assigned one of twenty possible labels depending on the original bulletin board where the document was posted.  Basic label frequency statistics are provided in Table~\ref{tab:20newsgroups}. The twenty document classes partition into seven super-classes -- referred to here as \textit{categories}.

In our experiments, the positive class is composed of the first four categories (alt., comp., misc., \&~rec.) while the remaining categories (sci., soc., \&~talk.) made up the negative class.  The positive class prior probability~($\prior$) is approximately~0.56.

\begin{table}[t]
  \centering
  \caption{20~newsgroup class \& category partition.  Each label's test set prior probability is listed.  For categories that contain more than one original label, the category's total prior probability is listed in the table margin.}\label{tab:20newsgroups}
  % Table atting based on: https://inf.ethz.ch/personal/markusp/teaching/guides/guide-tables.pdf
\begin{tabular}{@{}llll@{}}
  \toprule
  Class                      & Category & Label        & Prior \\\midrule
  \multirow{11}{*}{Positive} & alt.     & atheism      & \\\cline{2-4}
  & \multirow{5}{*}{comp.} & graphics            & \\
  & & os.ms-windows.misc  & \\
  & & sys.ibm.pc.hardware & \\
  & & sys.mac.hardware    & \\
  & & windows.x           & \\\cline{2-4}
  & misc.  & forsale     & \\\cline{2-4}
  &\multirow{4}{*}{rec.} & autos & \\
  & & motorcycles    & \\
  & & sport.baseball & \\
  & & sport.hockey   & \\\hline
  \multirow{9}{*}{Negative} & \multirow{4}{*}{sci.} & crypt & \\
  & & electronics & \\
  & & med          & \\
  & & space        & \\\cline{2-4}
  & soc.  & religion.christian & \\\cline{2-4}
  & \multirow{4}{*}{talk.} & politics.guns & \\
  & & politics.mideast & \\
  & & politics.misc    & \\
  & & religion.misc    & \\
  \bottomrule
\end{tabular}

\end{table}


\section{Experimental results}\label{sec:ExperimentalResults}

\begin{table}[t]
  \caption{Negative covariate shift accuracy results for the two classifier architectures.  Bias vector configurations are shown.  Unbiased PN~learning has no covariate shift and is the performance ceiling given the architecture and training set size.  For each experimental setup, the learner with the best performance is bolded.}\label{tab:ExperimentalResults}
  \begin{subtable}[t]{\textwidth}
    \centering
    \caption{End-to-end LSTM network}\label{tab:ExperimentalResults:Lstm}
    % Table atting based on: https://inf.ethz.ch/personal/markusp/teaching/guides/guide-tables.pdf
\begin{tabular}{@{}lllllllll@{}}
  \toprule
  \multicolumn{3}{c}{$\bntrain$ Bias} &    & \multicolumn{2}{c}{PN} &       &      \\\cmidrule{1-3}\cmidrule{5-6}
  sci.   & soc.   & talk.       & $\rho$   & Unbiased                          & Biased          & nnPU                             & PUbN \\\midrule
  100\%  & 0\%    & 0\%         & 0.21     & \multicolumn{1}{c}{$\uparrow$}    & 0.766           & \multicolumn{1}{c}{$\uparrow$}   & \textbf{0.870}\\
  0\%    & 0\%    & 100\%       & 0.17     & \multicolumn{1}{c}{0.883}         & 0.814           & \multicolumn{1}{c}{0.834}        & \textbf{0.846} \\
  10\%   & 50\%   & 40\%        & 0.1      & \multicolumn{1}{c}{$\downarrow$}  & \textbf{0.872}  & \multicolumn{1}{c}{$\downarrow$} & 0.822 \\
  \bottomrule
\end{tabular}

  \end{subtable}

  \begin{subtable}[t]{\textwidth}
    \centering
    \caption{ELMo preprocessed vectors}\label{tab:ExperimentalResults:Elmo}
    % Table atting based on: https://inf.ethz.ch/personal/markusp/teaching/guides/guide-tables.pdf
\begin{tabular}{@{}lllllllll@{}}
  \toprule
  \multicolumn{3}{c}{$\bntrain$ Bias} &    & \multicolumn{2}{c}{PN} &       &      \\\cmidrule{1-3}\cmidrule{5-6}
  sci.   & soc.   & talk.       & $\rho$   & Unbiased                          & Biased          & nnPU                             & PUbN \\\midrule
  100\%  & 0\%    & 0\%         & 0.21     & \multicolumn{1}{c}{$\uparrow$}    & 0.766           & \multicolumn{1}{c}{$\uparrow$}   & \textbf{0.870}\\
  0\%    & 0\%    & 100\%       & 0.17     & \multicolumn{1}{c}{0.883}         & 0.814           & \multicolumn{1}{c}{0.834}        & \textbf{0.846} \\
  10\%   & 50\%   & 40\%        & 0.1      & \multicolumn{1}{c}{$\downarrow$}  & \textbf{0.872}  & \multicolumn{1}{c}{$\downarrow$} & 0.822 \\
  \bottomrule
\end{tabular}

  \end{subtable}
\end{table}

\section{Conclusions}\label{sec:Conclusions}

This project empirically studied negative covariate shift's effects on 20~newsgroups classification.  The first key takeaway is that transfer learning's benefits compound when labeled training data is biased or limited.  Next, the most appropriate risk estimator will depend on the selection bias' exact characteristics; there is no ``one-size-fits-all'' solution.  This nuance is not discussed in Hsieh\etal's PUbN paper as they do not report biased~PN results.


\bibliographystyle{ieeetr}
\bibliography{bib/ref.bib}

\end{document}
