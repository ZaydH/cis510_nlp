\documentclass{article}

% if you need to pass options to natbib, use, e.g.:
%     \PassOptionsToPackage{numbers, compress}{natbib}
% before loading neurips_2019

% ready for submission
% \usepackage{neurips_2019}

% to compile a preprint version, e.g., for submission to arXiv, add add the
% [preprint] option:
  \usepackage[final,nonatbib]{neurips_2019}

% to compile a camera-ready version, add the [final] option, e.g.:
%     \usepackage[final]{neurips_2019}

% to avoid loading the natbib package, add option nonatbib:
%     \usepackage[nonatbib]{neurips_2019}

\usepackage[utf8]{inputenc} % allow utf-8 input
\usepackage[T1]{fontenc}    % use 8-bit T1 fonts
\usepackage{url}            % simple URL typesetting
\usepackage{booktabs}       % professional-quality tables
\usepackage{amsfonts}       % blackboard math symbols
\usepackage{nicefrac}       % compact symbols for 1/2, etc.
\usepackage{microtype}      % microtypography

../tex/global_macros.tex
\newcommand{\sourceCode}{https://github.com/ZaydH/covariate_shift_risk_estimation}

\newcommand{\X}{x}
\newcommand{\domainX}{\real^{d}}
\newcommand{\y}{y}
\newcommand{\domainY}{\set{{\pm}1}}

\newcommand{\pcls}{{+}1}
\newcommand{\ncls}{{-}1}

\newcommand{\params}{\theta}

\newcommand{\pDist}{p}
\newcommand{\joint}{\pDist_{\X\y}}
\newcommand{\marginal}{\pDist_{\X}}
\newcommand{\posterior}{\pDist(\y \vert \X)}
\newcommand{\pcond}{\pDist_{\textnormal{P}}}
\newcommand{\ncond}{\pDist_{\textnormal{N}}}
\newcommand{\bncond}{\pDist_{\textnormal{bN}}}
\newcommand{\prior}{\pi}

\newcommand{\latent}{s}
\newcommand{\trijoint}{\pDist_{\X\y\latent}}

\newcommand{\size}{n}
\newcommand{\train}{\mathcal{X}}
\newcommand{\trainVar}[1]{\train_{\textnormal{#1}}}
\newcommand{\utrain}{\mathscr{U}}
\newcommand{\ptrain}{\mathscr{P}}
\newcommand{\ntrain}{\mathscr{N}}
\newcommand{\bntrain}{\mathscr{B}}

\newcommand{\plabel}{\rho}

\newcommand{\risk}{R}
\newcommand{\baserisk}[2]{\risk_{#1}^{#2}}
\newcommand{\varrisk}[2]{\baserisk{\textnormal{#1}}{#2}}
\newcommand{\prisk}[1]{\varrisk{#1}{{+}}}
\newcommand{\nrisk}[1]{\varrisk{#1}{{-}}}
\newcommand{\smrisk}{\baserisk{\latent = \ncls}{-}}

\newcommand{\emprisk}{\hat{R}}
\newcommand{\evrisk}[2]{\hat{R}_{\textnormal{#1}}^{#2}}
\newcommand{\dec}{g}
\newcommand{\decX}{\dec\mathopen{}\left(\X\right)\mathclose{}}
\newcommand{\loss}{\ell}
% mathopen/mathclose used to remove extra free space around \left and \right keywords
\newcommand{\floss}[2]{\loss\mathopen{}\left(#1,#2\right)\mathclose{}}

\newcommand{\ypred}{\hat{y}}

\newcommand{\sigX}{\sigma(\X)}
\newcommand{\hsig}{\hat{\sigma}}
\newcommand{\hsigX}{\hat{\sigma}(\X)}
\newcommand{\xvar}[1]{\X_{\textnormal{#1}}}
\newcommand{\sigdiff}{\big(1 - \sigX\big)}


\usepackage{hyperref}       % hyperlinks

\title{Siamese-Based Autoencoders \\for Positive\-/Unlabeled Learning}

% The \author macro works with any number of authors. There are two commands
% used to separate the names and addresses of multiple authors: \And and \AND.
%
% Using \And between authors leaves it to LaTeX to determine where to break the
% lines. Using \AND forces a line break at that point. So, if LaTeX puts 3 of 4
% authors names on the first line, and the last on the second line, try using
% \AND instead of \And before the third author name.

\author{%
  Zayd S.\ Hammoudeh \\
  Department of Computer \& Information Science \\
  University of Oregon \\
  Eugene, OR 97403 \\
  \texttt{\href{mailto:zayd@cs.uoregon.edu}{zayd@cs.uoregon.edu}}
}

\begin{document}

\maketitle

\begin{abstract}
  Positive\-/unlabeled learning constructs a binary classifier using only positive-labeled and unlabeled examples -- there is no negative-labeled training examples. Highly-expressive learners like deep neural networks often overfit problems where labeled data is limited.  This paper presents our \textit{double decoder positive-unlabeled} (\toolname)~learner -- a novel autoencoder-based Siamese neural network architecture.  We propose a two-step training algorithm and accompanying set of loss functions that adapt the Siamese triplet loss to use only a single training example -- labeled or unlabeled.  Our empirical results show that our method achieves state-of-the-art performance on MNIST-variant datasets.
\end{abstract}

\documentclass[]{subfiles}

\begin{document}
\section{Introduction}\label{sec:Introduction}

Consider binary classification of text documents.  Each document is represented by two random variables: independent feature vector~${\X \in \domainX}$ and dependent label~${\y \in \domainY}$. The document population is generated from an unknown, joint probability distribution~${\joint \equiv \pDist(\X,\y)}$, which by Bayes' rule can be reformulated as

\begin{align}
  \underbrace{\pDist(\X,\y)}_{\joint} &= \pDist(\X) \posterior \label{eq:TrainDist} \\
           &= \underbrace{\pDist(\y = \pcls)}_{\prior} \underbrace{\pDist(\X \vert \y = \pcls)}_{\pcond} + \underbrace{\pDist(\y = \ncls)}_{1-\prior} \underbrace{\pDist(\X \vert \y = \ncls)}_{\ncond} \label{eq:Joint:Bayes}
\end{align}

\noindent
where $\prior$~is the positive\-/class prior while $\pcond$~and $\ncond$~are the positive and negative class-conditional distributions respectively.

Supervised binary classification's training set,~$\train$, consists of samples generated from~$\joint$. By Eq.~\eqref{eq:Joint:Bayes}, $\train$~partitions as ${\ptrain \sqcup \ntrain}$ where ${\ptrain \sim \pcond}$ are the positive-valued examples and ${\ntrain \sim \ncond}$ are negative-valued.  That is why supervised learning is alternatively referred to as \textit{positive\-/negative}~(PN) learning.

Ideal supervised learning often does not apply in practice.  Labeling may be expensive or difficult resulting in few labeled examples but numerous unlabeled ones.  Additionally, the labeled set may not be representative of~$\joint$.  For example, labeled examples may only characterize a small subset of $\joint$'s support, including there being no labeled data at all for one class.

One of the most common types of training set bias is \textit{covariate shift}. Define $\joint$~\eqref{eq:TrainDist} and ${\joint'}$~\eqref{eq:TestDist} as the training and test joint distributions respectively.  In covariate shit, posterior distribution,~${\posterior}$ is identical in both joint distributions. Rather, the distributions only differ in their respective marginals, i.e.,~${\marginal \equiv \pDist(\X)}$ and~$\marginal'$.~\cite{Huang:2006}  This paper assumes covariate shift for only one class --~the negative one.

\begin{equation}\label{eq:TestDist}
  \joint'(\X,\y) = \underbrace{\pDist'(\X)}_{\marginal'} \posterior
\end{equation}

The \textit{empirical risk minimization}~(ERM) training framework assumes that a low training-set expected risk correlates to low inference error. Training set bias, such as those aforementioned, undermine that assumption and can lead to large, unpredictable test set error rates.

Previous work has taken different approaches to overcome negative covariate shift's effects on text classification.  Li\etal~\cite{Li:2010} ignored the biased-negative training data entirely and learned a classifier using only the positive and unlabeled sets -- a paradigm known as \textit{\underline{p}ositive\-/\underline{u}nlabeled}~(PU) learning.  Fei \&~Liu~\cite{Fei:2015} took an alternate approach by ignoring any unlabeled data and learned a traditional supervised classifier using just the positive and biased-negative training examples.  Hsieh\etal~\cite{Hsieh:2018} combined the \underline{p}ositive, \underline{u}nlabeled, and \underline{b}iased \underline{n}egative~(PUbN) training sets into a single risk estimator that fits into the ERM~framework.

The primary contribution of this work is an empirical comparison of the PN, PU, and~PUbN estimators on 20~newsgroups~\cite{20newsgroups} classification under negative covariate shift.  The remainder of this document is structured as follows.  Section~\ref{sec:RiskEstimators} describes the PN, PU, and~PUbN risk estimation functions. Section~\ref{sec:Architectures} describes the two neural network architectures used in our experiments. Section~\ref{sec:20newsgroups} provides a brief overview of the 20~newsgroups dataset including how it is used in this paper. Section~\ref{sec:ExperimentalResults} describes our experimental setup, baselines, and results.  We provide brief concluding comments in Section~\ref{sec:Conclusions}.
\end{document}

\input{previous_work}
\input{siamese_networks}
\input{ddpu}
\section{Experimental results}\label{sec:ExperimentalResults}

We compare the performance of the PN, nnPU, and~PUbN risk estimators on both proposed neural architectures.  The experiments are modeled after the setup in~\cite{Hsieh:2018}.

For each labeled set (e.g.,~$\ptrain$, $\bntrain$,~$\ntrain$), 500~training examples were sampled from the corresponding conditional distribution.  The unlabeled training set consisted of 6,000~elements.

All learners were trained for 50 epochs. A disjoint validation set -- one-fifth the training set's size -- was used to select the best epoch. The training set's risk used the logistic loss.  The validation's set risk used the sigmoid loss to ensure no single example has an outsized impact on model selection.

The hyperparameters we tuned were learning rate ${\alpha \in \set{5 \cdot 10^{-3}, 10^{-3}, 5 \cdot 10^{-4}}}$, ${\tau \in \set{0.5, 0.7, 0.9}}$, and ${\gamma \in \set{0.1, 0.3, 0.5, 0.7, 0.9, 1}}$.  Each architecture and risk estimator pairing's optimal hyperparameter setting was selected using a grid search to find the minimum overall validation loss.

\subsection{Baselines}

The PN~learners serve as the performance baselines.  Unbiased~PN -- where ${\ptrain \sim \pcond}$ and ${\ntrain \sim \ncond}$ -- represents each architecture's performance ceiling given sizes~$\abs{\ptrain}$ and~$\abs{\ntrain}$.

Biased~PN follows an identical training procedure to its unbiased counterpart with the exception that ${\bntrain \sim \bncond}$~is used instead of $\ntrain$.  Since biased~PN is the simplest approach -- both theoretically and in implementation -- one would expect it would be handily outperformed by slower, more advanced risk estimators like~nnPU and~PUbN.

\subsection{Selection bias profiles}

Recall that under covariate shift, only the marginal distributions change.  As shown in Table~\eqref{tab:ExperimentalResults}, we achieve this by modifying the prior probabilities of each of the three negative categories.  The extent to which each category is biased can be determined by comparing Table~\ref{tab:ExperimentalResults}'s biases to Table~\ref{tab:20newsgroups}.

The selection biases we tested were of two variants -- each stressing different estimator limitations.  In the first two experiments, the biased negative set is drawn from only one negative category, i.e.,~sci.\ and~talk.\ respectively.  The second bias variant draws training examples from all negative categories but shifts the prior probabilities.

nnPU is unaffected by the biased negative set since it considers only~$\ptrain$ and~$\utrain$.  Unbiased~PN (by definition) does not consider~$\bntrain$ so it too is unaffected.  Therefore, only a single result per architecture is reported for those two risk estimators.

\subsection{Architecture comparison}

Tables~\ref{tab:ExperimentalResults:LSTM} and~\ref{tab:ExperimentalResults:ELMo} enumerate the test set (inductive) accuracy results for the~LSTM and preprocessed\-/ELMo architectures respectively.  Across all risk estimators and selection biases, the ELMo~preprocessed architecture performed better -- often substantially -- than the LSTM model as shown in Table~\ref{tab:ExperimentalResults:Comparison}.

ELMo's performance advantage most likely derives from its more extensive leveraging of transfer learning.  This indicates that when dealing with limited or biased labeled data, transfer learning's importance may compound.

We exclusively discuss the ELMo architecture going forward due to its vastly superior performance.

\subsection{Risk estimator comparison}

Table~\ref{tab:ExperimentalResults:ELMo}'s first two rows show that when the biased negative set is drawn from only a single negative category, PUbN had the best performance.  When the biased negative set was drawn from all negative categories, biased~PN had the best results.

To summarize formally, the most important factor in determining the best risk estimator is the extent to which the negative class-conditional distribution's support is covered.  Selecting all labeled examples from a single categories entails that a PN~learner has no guidance on how to classify examples from the remaining two negative categories.  That is why biased~PN performed so poorly in these tests.

In contrast when $\bntrain$~has labeled examples from all categories, biased~PN is still able to learn how to predict all categories and can outperform even classifiers that are provided information about the selection bias like~PUbN.

\begin{table}[t]
  \caption{20~newsgroups negative covariate shift test set accuracy results for the two classifier architectures and three bias configurations. Listed below each category name is its biased prior probability in that experiment. The corresponding labeling probability~$\plabel$ is also provided. The best performing learners are bolded.}\label{tab:ExperimentalResults}
  \begin{subtable}[t]{\textwidth}
    \centering
    \caption{End-to-end LSTM architecture results. Due to the longer LSTM training time and limited computational resoruces, the grid search used only a single trial per hyperparameter setting.}\label{tab:ExperimentalResults:LSTM}
    % Table atting based on: https://inf.ethz.ch/personal/markusp/teaching/guides/guide-tables.pdf
\begin{tabular}{@{}lllllllll@{}}
  \toprule
  \multicolumn{3}{c}{$\bntrain$ Bias} &    & \multicolumn{2}{c}{PN} &       &      \\\cmidrule{1-3}\cmidrule{5-6}
  sci.   & soc.   & talk.       & $\rho$   & Unbiased                          & Biased          & nnPU                             & PUbN \\\midrule
  100\%  & 0\%    & 0\%         & 0.21     & \multicolumn{1}{c}{$\uparrow$}    & 0.640           & \multicolumn{1}{c}{$\uparrow$}   & \textbf{0.677}\\
  0\%    & 0\%    & 100\%       & 0.17     & \multicolumn{1}{c}{0.704}         & \textbf{0.768}  & \multicolumn{1}{c}{0.604}        & 0.675 \\
  10\%   & 50\%   & 40\%        & 0.1      & \multicolumn{1}{c}{$\downarrow$}  & \textbf{0.761}  & \multicolumn{1}{c}{$\downarrow$} & 0.601 \\
  \bottomrule
\end{tabular}

  \end{subtable}

  \begin{subtable}[t]{\textwidth}
    \centering
    \caption{Preprocessed\-/ELMo architecture results averaged across 10~independent trials}\label{tab:ExperimentalResults:ELMo}
    % Table atting based on: https://inf.ethz.ch/personal/markusp/teaching/guides/guide-tables.pdf
\begin{tabular}{@{}lllllllll@{}}
  \toprule
  \multicolumn{3}{c}{$\bntrain$ Bias} &    & \multicolumn{2}{c}{PN} &       &      \\\cmidrule{1-3}\cmidrule{5-6}
  sci.   & soc.   & talk.       &    & Unbiased                          & Biased          & nnPU                             & PUbN \\\midrule
  100\%  & 0\%    & 0\%         &    & \multicolumn{1}{c}{$\uparrow$}    & 0.766           & \multicolumn{1}{c}{$\uparrow$}   & \textbf{0.870}\\
  0\%    & 0\%    & 100\%       &    & \multicolumn{1}{c}{0.883}         & 0.814           & \multicolumn{1}{c}{0.834}        & \textbf{0.846} \\
  10\%   & 50\%   & 40\%        &    & \multicolumn{1}{c}{$\downarrow$}  & \textbf{0.872}  & \multicolumn{1}{c}{$\downarrow$} & 0.822 \\
  \bottomrule
\end{tabular}

  \end{subtable}

  \begin{subtable}[t]{\textwidth}
    \centering
    \caption{Performance difference between the preprocessed-ELMo \& LSTM results}\label{tab:ExperimentalResults:Comparison}
    % Table atting based on: https://inf.ethz.ch/personal/markusp/teaching/guides/guide-tables.pdf
\begin{tabular}{@{}lllllllll@{}}
  \toprule
  \multicolumn{3}{c}{$\bntrain$ Bias} &    & \multicolumn{2}{c}{PN} &       &      \\\cmidrule{1-3}\cmidrule{5-6}
  sci.   & soc.   & talk.       &   & Unbiased                          & Biased          & nnPU                             & PUbN \\\midrule
  100\%  & 0\%    & 0\%         &   & \multicolumn{1}{c}{$\uparrow$}    & 0.126           & \multicolumn{1}{c}{$\uparrow$}   & 0.193 \\
  0\%    & 0\%    & 100\%       &   & \multicolumn{1}{c}{0.179}         & 0.046           & \multicolumn{1}{c}{0.230}        & 0.171 \\
  10\%   & 50\%   & 40\%        &   & \multicolumn{1}{c}{$\downarrow$}  & 0.111           & \multicolumn{1}{c}{$\downarrow$} & 0.221 \\
  \bottomrule
\end{tabular}

  \end{subtable}
\end{table}

\section{Conclusions}\label{sec:Conclusions}

This project empirically studied negative covariate shift's effects on 20~newsgroups classification.  The first key takeaway is that transfer learning's benefits compound when labeled training data is biased or limited.  Next, the most appropriate risk estimator will depend on the selection bias' exact characteristics; there is no ``one-size-fits-all'' solution.  This nuance is not discussed in Hsieh\etal's PUbN paper as they do not report biased~PN results.


\bibliographystyle{ieeetr}
\bibliography{bib/ref.bib}

\end{document}
