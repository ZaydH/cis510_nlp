\documentclass{article}

% if you need to pass options to natbib, use, e.g.:
%     \PassOptionsToPackage{numbers, compress}{natbib}
% before loading neurips_2019

% ready for submission
% \usepackage{neurips_2019}

% to compile a preprint version, e.g., for submission to arXiv, add add the
% [preprint] option:
  \usepackage[final,nonatbib]{neurips_2019}

% to compile a camera-ready version, add the [final] option, e.g.:
%     \usepackage[final]{neurips_2019}

% to avoid loading the natbib package, add option nonatbib:
%     \usepackage[nonatbib]{neurips_2019}

\usepackage[utf8]{inputenc} % allow utf-8 input
\usepackage[T1]{fontenc}    % use 8-bit T1 fonts
\usepackage{url}            % simple URL typesetting
\usepackage{booktabs}       % professional-quality tables
\usepackage{amsfonts}       % blackboard math symbols
\usepackage{nicefrac}       % compact symbols for 1/2, etc.
\usepackage{microtype}      % microtypography

\newcommand{\etal}{et~al.}
\newcommand{\elkan}{Elkan \&~Noto}

% Used for including standalone docs
\usepackage{standalone}

\newcommand{\transpose}{^{\text{T}}}

% Imported via UltiSnips
% Unbreakable dash:
%  Words hyphened with these dashes can also be broken at other positions than the dash
%    \-/ hyphen
%    \-- en-dash
%    \--- em-dash
%    extdash unbreakable dashes
%
%  No line breaks possible at the hyphen
%    \=/ hyphen
%    \== en-dash
%    \=== em-dash
\usepackage[shortcuts]{extdash}

% Imported via UltiSnips
\usepackage[dvipsnames]{xcolor}
\newcommand{\colortext}[2]{{\color{#1} #2}}
\newcommand{\red}[1]{\colortext{red}{#1}}
\newcommand{\blue}[1]{\colortext{blue}{#1}}
\newcommand{\green}[1]{\colortext{green}{#1}}

% Imported via UltiSnips
\usepackage{amsmath}
\DeclareMathOperator*{\argmax}{arg\,max}
\DeclareMathOperator*{\argmin}{arg\,min}
\usepackage{amsfonts}  % Used for \mathbb and \mathcal
\usepackage{amssymb}

% Imported via UltiSnips
\usepackage{mathtools} % for "\DeclarePairedDelimiter" macro
% \swapifbranches changes unstarred paired delimiters to starred and
% vice versa.  This means by default, paired delimiters have the star.
\usepackage{etoolbox}
\newcommand\swapifbranches[3]{#1{#3}{#2}}
\makeatletter
\MHInternalSyntaxOn
\patchcmd{\DeclarePairedDelimiter}{\@ifstar}{\swapifbranches\@ifstar}{}{}
\MHInternalSyntaxOff
\makeatother
% Place after swap to ensure swap star
\DeclarePairedDelimiter{\sbrack}{\lbrack}{\rbrack}
\DeclarePairedDelimiter{\floor}{\lfloor}{\rfloor}
\DeclarePairedDelimiter{\ceil}{\lceil}{\rceil}
\DeclarePairedDelimiter{\abs}{\lvert}{\rvert}
\DeclarePairedDelimiter{\norm}{\lVert}{\rVert}
\usepackage{bm}
\DeclarePairedDelimiterX\set[1]\lbrace\rbrace{#1}
\DeclarePairedDelimiterX\setbuild[2]\lbrace\rbrace{#1 \bm: #2}
\newcommand{\ints}[1]{{\sbrack{#1}}}
\newcommand{\func}[3]{{#1:#2\rightarrow#3}}
\newcommand{\defeq}{\stackrel{\mathclap{\mbox{\tiny def}}}{=}}

% Imported via UltiSnips
\usepackage{multirow}
\usepackage{booktabs}

% Imported via UltiSnips
\usepackage{tikz}
\usetikzlibrary{arrows,decorations.markings,shadows,positioning,calc,backgrounds,shapes}

% Imported via UltiSnips
\usepackage[noend]{algpseudocode}
\usepackage[Algorithm,ruled]{algorithm}
\algnewcommand\algorithmicforeach{\textbf{for each}}
\algdef{S}[FOR]{ForEach}[1]{\algorithmicforeach\ #1\ \algorithmicdo}
\newcommand{\algin}[1]{\hspace*{\algorithmicindent} \textbf{Input} #1\\}
\newcommand{\algout}[1]{\hspace*{\algorithmicindent} \textbf{Output} #1\\}

\newcommand{\toolname}{ddPU}

\newcommand{\Repel}{Repellent}
\newcommand{\repel}{\MakeLowercase{\Repel}}

\newcommand{\X}{x}
\newcommand{\domain}{\real^{d}}
\newcommand{\eX}[1]{\X_{(#1)}}
\newcommand{\xA}{\eX{A}}
\newcommand{\xP}{\eX{P}}
\newcommand{\xN}{\eX{N}}

\newcommand{\y}{y}
\newcommand{\labels}{\set{{\pm}1}}
\newcommand{\pCls}{{+}1}
\newcommand{\nCls}{{-}1}

\newcommand{\latent}{s}
\newcommand{\params}{\theta}

\newcommand{\pDist}{p}
\newcommand{\joint}{\pDist_{XY}}
\newcommand{\marginal}{\pDist_{X}}

\newcommand{\size}{n}
\newcommand{\train}{\mathcal{X}}
\newcommand{\trainVar}[1]{\train_{\textnormal{#1}}}
\newcommand{\unlabel}{\trainVar{U}}
\newcommand{\pos}{\trainVar{P}}
% Batch specific naming
\newcommand{\cnt}{i}
\newcommand{\unlI}[1]{#1{\mathcal{X}}_{\textnormal{U}}^{(\cnt)}}
\newcommand{\posI}[1]{#1{\mathcal{X}}_{\textnormal{P}}^{(\cnt)}}


% Siamese Network Related Macros
\newcommand{\margin}{\alpha}
\newcommand{\SiamDim}{m}
\newcommand{\SiamFunc}{f}
\newcommand{\distMetric}{\delta}
\newcommand{\siamDist}[3]{\norm{#1\left(#2\right) - #1\left(#3\right)}}
\newcommand{\lTrip}{\mathcal{L}_{\text{Triplet}}}

% Function names for the DeepPU architecture
\newcommand{\fPU}{g}
\newcommand{\fPUenc}{\fPU_{\text{enc}}}
\newcommand{\fPUp}{\fPU_{p}}
\newcommand{\fPUn}{\fPU_{n}}


\usepackage{hyperref}       % hyperlinks

\title{Siamese-Based Autoencoders \\for Positive\-/Unlabeled Learning}

% The \author macro works with any number of authors. There are two commands
% used to separate the names and addresses of multiple authors: \And and \AND.
%
% Using \And between authors leaves it to LaTeX to determine where to break the
% lines. Using \AND forces a line break at that point. So, if LaTeX puts 3 of 4
% authors names on the first line, and the last on the second line, try using
% \AND instead of \And before the third author name.

\author{%
  Zayd S.\ Hammoudeh \\
  Department of Computer \& Information Science \\
  University of Oregon \\
  Eugene, OR 97403 \\
  \texttt{\href{mailto:zayd@cs.uoregon.edu}{zayd@cs.uoregon.edu}}
}

\begin{document}

\maketitle

\begin{abstract}
  Positive\-/unlabeled learning constructs a binary classifier using only positive-labeled and unlabeled examples -- there is no negative-labeled training examples. Highly-expressive learners like deep neural networks often overfit problems where labeled data is limited.  This paper presents our \textit{double decoder positive-unlabeled} (\toolname)~learner -- a novel autoencoder-based Siamese neural network architecture.  We propose a two-step training algorithm and accompanying set of loss functions that adapt the Siamese triplet loss to use only a single training example -- labeled or unlabeled.  Our empirical results show that our method achieves state-of-the-art performance on MNIST-variant datasets.
\end{abstract}

\documentclass[]{subfiles}

\begin{document}
\section{Introduction}\label{sec:Introduction}

Consider binary classification of text documents.  Each document is represented by two random variables: independent feature vector~${\X \in \domainX}$ and dependent label~${\y \in \domainY}$. The document population is generated from an unknown, joint probability distribution~${\joint(\X,\y)}$.  By Bayes' rule, the joint distribution decomposes as

\begin{align}
    \joint(\X,\y) &= \marginal(\X) \posterior \label{eq:TrainDist} \\
                  &= \pDist(\y = \pcls) \pDist(\X \vert \y = \pcls) + \pDist(\y = \ncls) \pDist(\X \vert \y = \ncls) \nonumber\\
                  &\equiv \prior \pcond(\X) + (1 - \prior) \ncond(\X) \label{eq:Joint:Bayes}
\end{align}

\noindent
where $\prior$~is the positive class prior probability while $\pcond$~and $\ncond$~are the positive and negative class-conditional distributions respectively.

Supervised binary classification's training set,~$\train$, traditionally consists of $\size$~independent samples from Eq.~\eqref{eq:Joint:Bayes}.  Therefore, the training set partitions as ${\train = \ptrain \sqcup \ntrain}$ where ${\ptrain \sim \pcond}$ are the positive-valued examples and ${\ntrain \sim \ncond}$ are negative-valued.  For that reason, this training paradigm is often referred to as \textit{positive-negative}~(PN) learning.

This idealized supervised training model often does not apply in practice.  Training set labeling can be expensive or difficult meaning there are few labeled examples but numerous unlabeled ones.  Additionally, the labeled set may not be representative of~$\joint$.  For example, the training examples only characterizes a small subset of $\joint$'s support, including where there is no labeled data at all for one class.

The \textit{empirical risk minimization} training framework assumes that a low training set expected risk correlates to low inference error. Training set bias, such as those aforementioned, undermine that assumption, and can lead to large, unpredictable test set error rates.

One of the most common types of training set bias is \textit{covariate shift}. Define $\joint$~\eqref{eq:TrainDist} and ${\joint'}$~\eqref{eq:TestDist} as the training and set joint distributions respectively.  Posterior distribution,~${\posterior}$ is identical in both cases; rather, they only differ in their marginal distributions, i.e.,~$\marginal$ and~${\marginal'}$.~\cite{Huang:2006}  This paper assumes covariate shift for only one class, i.e.,~the negative one.

\begin{equation}\label{eq:TestDist}
    \joint'(\X,\y) = \marginal'(\X) \posterior
\end{equation}

Previous work has taken different routes to overcome negative covariate shift's effects on text classification.  Li\etal~\cite{Li:2010} ignored the biased negative training data entirely and attempted to learn a classifier using only the positive and unlabeled sets -- an approach known as \textit{positive\-/unlabeled}~(PU) learning.  Fei \&~Liu~\cite{Fei:2015} took the alternate approach of ignoring any unlabeled data and learned a traditional supervised classifier using just the positive and biased-negative training examples.  Hsieh\etal~\cite{Hsieh:2018} jointly used the \underline{p}ositive, \underline{u}nlabeled, and \underline{b}iased \underline{n}egative~(PUbN) in a risk estimator that works with standard empirical risk minimization.

The primary contribution of this work is an empirical comparison of PN, PU, and~PUbN classifiers on the 20~newsgroups dataset~\cite{20newsgroups} under negative covariate shift.  The remainder of this document is structured as follows.  Section~\ref{sec:RiskEstimators} describes the PN, PU, and~PUBN risk estimation functions. Section~\ref{sec:Architectures} describes the two neural network architectures used. Section~\ref{sec:20newsgroups} provides a brief overview of the 20~newsgroups dataset including how it is used in this paper. Section~\ref{sec:ExperimentalResults} describes our experimental setup, baselines, and results.  We provide brief concluding comments in Section~\ref{sec:Conclusions}.
\end{document}

\input{previous_work}
\input{siamese_networks}
\input{ddpu}
\section{Experimental results}\label{sec:ExperimentalResults}

\begin{table}[t]
  \caption{Negative covariate shift accuracy results for the two classifier architectures.  Bias vector configurations are shown.  Unbiased PN~learning has no covariate shift and is the performance ceiling given the architecture and training set size.  For each experimental setup, the learner with the best performance is bolded.}\label{tab:ExperimentalResults}
  \begin{subtable}[t]{\textwidth}
    \centering
    \caption{End-to-end LSTM network}\label{tab:ExperimentalResults:Lstm}
    % Table atting based on: https://inf.ethz.ch/personal/markusp/teaching/guides/guide-tables.pdf
\begin{tabular}{@{}lllllllll@{}}
  \toprule
  \multicolumn{3}{c}{$\bntrain$ Bias} &    & \multicolumn{2}{c}{PN} &       &      \\\cmidrule{1-3}\cmidrule{5-6}
  sci.   & soc.   & talk.       & $\rho$   & Unbiased                          & Biased          & nnPU                             & PUbN \\\midrule
  100\%  & 0\%    & 0\%         & 0.21     & \multicolumn{1}{c}{$\uparrow$}    & 0.766           & \multicolumn{1}{c}{$\uparrow$}   & \textbf{0.870}\\
  0\%    & 0\%    & 100\%       & 0.17     & \multicolumn{1}{c}{0.883}         & 0.814           & \multicolumn{1}{c}{0.834}        & \textbf{0.846} \\
  10\%   & 50\%   & 40\%        & 0.1      & \multicolumn{1}{c}{$\downarrow$}  & \textbf{0.872}  & \multicolumn{1}{c}{$\downarrow$} & 0.822 \\
  \bottomrule
\end{tabular}

  \end{subtable}

  \begin{subtable}[t]{\textwidth}
    \centering
    \caption{ELMo preprocessed vectors}\label{tab:ExperimentalResults:Elmo}
    % Table atting based on: https://inf.ethz.ch/personal/markusp/teaching/guides/guide-tables.pdf
\begin{tabular}{@{}lllllllll@{}}
  \toprule
  \multicolumn{3}{c}{$\bntrain$ Bias} &    & \multicolumn{2}{c}{PN} &       &      \\\cmidrule{1-3}\cmidrule{5-6}
  sci.   & soc.   & talk.       & $\rho$   & Unbiased                          & Biased          & nnPU                             & PUbN \\\midrule
  100\%  & 0\%    & 0\%         & 0.21     & \multicolumn{1}{c}{$\uparrow$}    & 0.766           & \multicolumn{1}{c}{$\uparrow$}   & \textbf{0.870}\\
  0\%    & 0\%    & 100\%       & 0.17     & \multicolumn{1}{c}{0.883}         & 0.814           & \multicolumn{1}{c}{0.834}        & \textbf{0.846} \\
  10\%   & 50\%   & 40\%        & 0.1      & \multicolumn{1}{c}{$\downarrow$}  & \textbf{0.872}  & \multicolumn{1}{c}{$\downarrow$} & 0.822 \\
  \bottomrule
\end{tabular}

  \end{subtable}
\end{table}

\section{Conclusions}\label{sec:Conclusions}

This project empirically studied negative covariate shift's effects on 20~newsgroups classification.  The first key takeaway is that transfer learning's benefits compound when labeled training data is biased or limited.  Next, the most appropriate risk estimator will depend on the selection bias' exact characteristics; there is no ``one-size-fits-all'' solution.  This nuance is not discussed in Hsieh\etal's PUbN paper as they do not report biased~PN results.


\bibliographystyle{ieeetr}
\bibliography{bib/ref.bib}

\end{document}
