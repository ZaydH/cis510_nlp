\documentclass[10pt]{article}
\usepackage[margin=1in]{geometry}

../tex/global_macros.tex
\newcommand{\sourceCode}{https://github.com/ZaydH/covariate_shift_risk_estimation}

\newcommand{\X}{x}
\newcommand{\domainX}{\real^{d}}
\newcommand{\y}{y}
\newcommand{\domainY}{\set{{\pm}1}}

\newcommand{\pcls}{{+}1}
\newcommand{\ncls}{{-}1}

\newcommand{\params}{\theta}

\newcommand{\pDist}{p}
\newcommand{\joint}{\pDist_{\X\y}}
\newcommand{\marginal}{\pDist_{\X}}
\newcommand{\posterior}{\pDist(\y \vert \X)}
\newcommand{\pcond}{\pDist_{\textnormal{P}}}
\newcommand{\ncond}{\pDist_{\textnormal{N}}}
\newcommand{\bncond}{\pDist_{\textnormal{bN}}}
\newcommand{\prior}{\pi}

\newcommand{\latent}{s}
\newcommand{\trijoint}{\pDist_{\X\y\latent}}

\newcommand{\size}{n}
\newcommand{\train}{\mathcal{X}}
\newcommand{\trainVar}[1]{\train_{\textnormal{#1}}}
\newcommand{\utrain}{\mathscr{U}}
\newcommand{\ptrain}{\mathscr{P}}
\newcommand{\ntrain}{\mathscr{N}}
\newcommand{\bntrain}{\mathscr{B}}

\newcommand{\plabel}{\rho}

\newcommand{\risk}{R}
\newcommand{\baserisk}[2]{\risk_{#1}^{#2}}
\newcommand{\varrisk}[2]{\baserisk{\textnormal{#1}}{#2}}
\newcommand{\prisk}[1]{\varrisk{#1}{{+}}}
\newcommand{\nrisk}[1]{\varrisk{#1}{{-}}}
\newcommand{\smrisk}{\baserisk{\latent = \ncls}{-}}

\newcommand{\emprisk}{\hat{R}}
\newcommand{\evrisk}[2]{\hat{R}_{\textnormal{#1}}^{#2}}
\newcommand{\dec}{g}
\newcommand{\decX}{\dec\mathopen{}\left(\X\right)\mathclose{}}
\newcommand{\loss}{\ell}
% mathopen/mathclose used to remove extra free space around \left and \right keywords
\newcommand{\floss}[2]{\loss\mathopen{}\left(#1,#2\right)\mathclose{}}

\newcommand{\ypred}{\hat{y}}

\newcommand{\sigX}{\sigma(\X)}
\newcommand{\hsig}{\hat{\sigma}}
\newcommand{\hsigX}{\hat{\sigma}(\X)}
\newcommand{\xvar}[1]{\X_{\textnormal{#1}}}
\newcommand{\sigdiff}{\big(1 - \sigX\big)}


% Enable (uncolored) cross-reference hyperlinks
% Should always be last package loaded.
% See: https://tex.stackexchange.com/questions/103123/links-do-not-lead-to-right-pages
\usepackage[colorlinks=false]{hyperref}
\begin{document}
  \begin{center}
    {\Large \textbf{Text Classification under Negative Covariate Shift}}

    \vspace{6pt}{\large Project Proposal}

    \vspace{6pt}Zayd Hammoudeh
  \end{center}

  Consider the task of binary (supervised) classification.  Each sample is a tuple of feature vector~${\X \in \domain}$ and label~${\y \in \lbls}$.  Samples are generated from unknown, joint probability distribution~$\joint$.  Given hypothesis class~${\hypoCls=\setbuild{\dec}{\func{\dec}{\domain}{\real}}}$ and loss function~$\func{\loss}{\real \times \lbls}{\real}$, the ideal classifier is:

  \begin{equation}\label{eq:IdealClassifier}
    \dec\opt = \argmin_{\dec \in \hypoCls} \mathbb{E}_{(\X, \y) \sim \joint} \sbrack{\fLoss{\fDec{\X}, \y}}\text{.}
  \end{equation}

  Since $\joint$~is unknown, Eq.~\eqref{eq:IdealClassifier} cannot be determined exactly.  \textit{Empirical risk minimization}, defined in Eq.~\eqref{eq:ERM}, is used in practice for selection of the optimal learner,~${\dec\opt}$. Observe that the true expected loss is replaced with an empirical estimate using i.i.d.\ training set~${\train \sim \joint}$.

  \begin{equation}\label{eq:ERM}
    \dec\opt = \argmin_{\dec \in \hypoCls} \sum_{(\X_i,\y_i) \in \train} \fLoss{\fDec{\X_i}, \y_i}
  \end{equation}

  There are many real-world applications where $\train$~is not an i.i.d.\ sample. For example, given a huge (unlabeled) news corpus, a particular user may only be interested in a single topic,~$\interest$.  There are numerous irrelevant news topics,~${\irrel_1,\ldots,\irrel_n}$ where~$n$ is huge.  This user could label a few relevant articles she manually found as positive~($\pCls$). She could also label any irrelevant articles she encountered negative. Such an ad-hoc labeling mechanism generally induces \textit{covariate shift} (i.e.,~train and test marginal distributions) for the negative class. For instance, the user may fail to label documents from each negative topic~$\irrel_i$. Similarly, the training set's negative topic prior probabilities will not match the corresponding true priors. Constructing a classifier given negative covariate shift is know as \textit{positive, unlabeled, biased negative} (PUbN)~\textit{learning}.

  There are three common approaches to PUbN learning:

  \begin{enumerate}
    \item Supervised (PN -- Positive-Negative) Learning: Ignores the bias in the negative training examples.  Serves as a baseline of comparison and quantifies the effect of any correction.

    \item Positive\-/Unlabeled (PU) Learning: Premised on the idea that using the biased negative training examples can be inherently deleterious.  It constructs a binary classifier strictly using the positive examples and the residual unlabeled corpus.  An example where this approach was used is~\cite{Li:2010}.

    \item Positive, Unlabeled, biased Negative (PUbN) Learning: Constructs a classifier using some \textit{a~priori} knowledge of the negative training set's bias.  Similar to~\cite{Hsieh:2019}, this project assumes knowledge of the negative training sets prior probabilities.
  \end{enumerate}

  \paragraph{Goal} This project will explore negative covariate shift's effect on news topic classification.  The experiments in this section are inspired by Fei \&~Liu~\cite{Fei:2015} and Hsieh\etal~\cite{Hsieh:2019}.

  \paragraph{Experimental Setup} This project will use the 20~newsgroups dataset~\cite{20Newsgroups}.  The dataset statistics are shown in Table~\ref{tab:DatasetComparison}.  We will use the latest (cleaned) version of the dataset published by Jason Rennie where duplicate and empty documents were removed.

  \begin{table}[b]
    \centering
    \caption{Dataset statistics for 20~Newsgroups}\label{tab:DatasetComparison}
    \begin{tabular}{|c|c|c|}
      \hline
      \#~Classes     & 20                     \\\hline
      Train Size     & 13,180                 \\\hline
      Test Size      & 5,648                  \\\hline
      Test Set Prior & Nearly uniform         \\\hline
    \end{tabular}
  \end{table}


  When constructing the experimental datasets, one topic will be selected as the positive class and all remaining~(19) topics will form joint form the negative class.  The experiments will be divided into three categories:

  \begin{itemize}
    \item \textit{In-Training}: All negative subclasses (${\irrel_1,\ldots,\irrel_n}$) are present in both the training and test sets.  The only difference is their train/test marginal distributions.

    \item \textit{Out-of-Training}: Some negative subclasses are only present in the test set. Beside that difference, the train and test marginal distributions are identical.

    \item \textit{Combined}: Synthesis of in-training and out-of-training.  Bias includes missing training set subclasses and a marginal distribution shifts  Most challenging case.
  \end{itemize}

  \noindent
  This procedure will be repeated with different topics as the positive class to ensure as representative of an experiment set as possible.

  \paragraph{Implementation Plan} The PN, PU, and PUbN implementations will share the same standard neural network architecture implemented in PyTorch.  There will be an embedding of some kind (at minimum a word embedding but potentially also character as well), a bidirectional (possibly stacked) RNN, and an output feed forward.

  Where the three learning classes differ is the risk estimator used in gradient descent training.  PN~will use standard logistic loss.  PU~will use the nnPU risk estimator by Kiryo\etal\ in~\cite{Kiryo:2017}.  PUbN will use~\cite{Hsieh:2019}'s risk estimator.

  \bibliographystyle{ieeetr}
  \bibliography{bib/ref.bib}
\end{document}
