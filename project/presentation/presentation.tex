\documentclass[11pt,dvipsnames,usenames,aspectratio=169]{beamer}  % Add handout to options to disable overlays

% For more themes, color themes and font themes, see:
% http://deic.uab.es/~iblanes/beamer_gallery/index_by_theme.html
%
\mode<presentation>
{%
  \usetheme{CambridgeUS}    % or try default, Darmstadt, Warsaw, ...
  \usecolortheme{whale}     % or try albatross, beaver, crane, ...
  \usefonttheme{serif}          % or try default, structurebold, ...
  % \usefonttheme[onlymath]{serif}
  % \setbeamertemplate{navigation symbols}{}
  % \setbeamercovered{transparent}

  \setbeamercolor{title}{fg=white}
  \setbeamerfont{title}{series=\bfseries}
  \setbeamercolor{frametitle}{fg=black}
  \setbeamerfont{frametitle}{series=\bfseries}

  \setbeamercolor{section in head/foot}{fg=white}
  \setbeamerfont{section in head/foot}{series=\bfseries}
  \setbeamercolor{subsection in head/foot}{fg=white}
  \setbeamerfont{subsection in head/foot}{series=\bfseries}
  \setbeamercolor{author in head/foot}{fg=white}
  \setbeamerfont{author in head/foot}{series=\bfseries}
  \setbeamercolor{title in head/foot}{fg=white}
  \setbeamerfont{title in head/foot}{series=\bfseries}

  \setbeamercolor{block title}{use=structure,fg=white,bg=title in head/foot.bg}
  \setbeamerfont{block title}{series=\bfseries}
  \setbeamercolor{block body}{use=structure,fg=black,bg=black!3!white}
}

% Support graying out frame elements
\newcommand{\FrameOpague}{\setbeamercovered{again covered={\opaqueness<1->{40}}}}
% Transition slide
\newcommand{\transitionFrame}[1]{%
{%
  \begin{frame}[plain,noframenumbering]{}{} % the plain option removes the sidebar and header from the title page
    \setbeamertemplate{final page}[text]{\Large \textbf{#1}}
    \usebeamertemplate{final page}
  \end{frame}}
}

\newcommand{\sourceCode}{https://github.com/ZaydH/covariate_shift_risk_estimation}

\newcommand{\X}{x}
\newcommand{\domainX}{\real^{d}}
\newcommand{\y}{y}
\newcommand{\domainY}{\set{{\pm}1}}

\newcommand{\pcls}{{+}1}
\newcommand{\ncls}{{-}1}

\newcommand{\params}{\theta}

\newcommand{\pDist}{p}
\newcommand{\joint}{\pDist_{\X\y}}
\newcommand{\marginal}{\pDist_{\X}}
\newcommand{\posterior}{\pDist(\y \vert \X)}
\newcommand{\pcond}{\pDist_{\textnormal{P}}}
\newcommand{\ncond}{\pDist_{\textnormal{N}}}
\newcommand{\bncond}{\pDist_{\textnormal{bN}}}
\newcommand{\prior}{\pi}

\newcommand{\latent}{s}
\newcommand{\trijoint}{\pDist_{\X\y\latent}}

\newcommand{\size}{n}
\newcommand{\train}{\mathcal{X}}
\newcommand{\trainVar}[1]{\train_{\textnormal{#1}}}
\newcommand{\utrain}{\mathscr{U}}
\newcommand{\ptrain}{\mathscr{P}}
\newcommand{\ntrain}{\mathscr{N}}
\newcommand{\bntrain}{\mathscr{B}}

\newcommand{\plabel}{\rho}

\newcommand{\risk}{R}
\newcommand{\baserisk}[2]{\risk_{#1}^{#2}}
\newcommand{\varrisk}[2]{\baserisk{\textnormal{#1}}{#2}}
\newcommand{\prisk}[1]{\varrisk{#1}{{+}}}
\newcommand{\nrisk}[1]{\varrisk{#1}{{-}}}
\newcommand{\smrisk}{\baserisk{\latent = \ncls}{-}}

\newcommand{\emprisk}{\hat{R}}
\newcommand{\evrisk}[2]{\hat{R}_{\textnormal{#1}}^{#2}}
\newcommand{\dec}{g}
\newcommand{\decX}{\dec\mathopen{}\left(\X\right)\mathclose{}}
\newcommand{\loss}{\ell}
% mathopen/mathclose used to remove extra free space around \left and \right keywords
\newcommand{\floss}[2]{\loss\mathopen{}\left(#1,#2\right)\mathclose{}}

\newcommand{\ypred}{\hat{y}}

\newcommand{\sigX}{\sigma(\X)}
\newcommand{\hsig}{\hat{\sigma}}
\newcommand{\hsigX}{\hat{\sigma}(\X)}
\newcommand{\xvar}[1]{\X_{\textnormal{#1}}}
\newcommand{\sigdiff}{\big(1 - \sigX\big)}

../tex/global_macros.tex

\usepackage{color}
\renewcommand{\blue}[1]{{\color{Blue} #1}}
\renewcommand{\red}[1]{{\color{red} #1}}
\renewcommand{\green}[1]{{\color{ForestGreen} #1}}

% Here's where the presentation starts, with the info for the title slide
\title[Negative Covariate Shift]{Address Negative Covariate Shift \\ on 20~Newsgroups Classification}
\author[Zayd Hammoudeh]{%
  \href{mailto:zayd@cs.uoregon.edu}{\textbf{Zayd Hammoudeh}}\inst{1}  % \textsuperscript{(\Letter)}
  % \and
  % \href{mailto:lowd@cs.uoregon.edu}{Daniel Lowd}\inst{1}
}

\institute[Univ.\ Oregon]{%
  \textsuperscript{1}\textbf{University of Oregon}\\
  Eugene, OR, USA\\
  \texttt{\href{mailto:zayd@cs.uoregon.edu}{zayd@cs.uoregon.edu}}
  % \texttt{{zayd, lowd}@ucsc.edu}
}
\date{\today}


\begin{document}

\begin{frame}
  \titlepage
\end{frame}

\begin{frame}{Problem Setting}
  \begin{itemize}
    \item Construct a supervised, binary text document classifier
  \end{itemize}
\end{frame}

\begin{frame}{Classifier Architectures}
  \onslide<+->{We considered two deep architectures:}
  \vfill
  \begin{itemize}[<+->]
    \setlength{\itemsep}{24pt}
    \item \blue{\textbf{``Classic'' RNN Classifier}}: Tokenize sentence, encode with embedding matrix, classify with RNN + feedforward

    \item \blue{\textbf{ELMo Preprocess}}: Encode entire document into a static representation with ELMo (a ``document embedding'').  Train a feedforward to classify the static vectors.
  \end{itemize}
  \vfill
  \onslide<+->{\green{\textbf{How Can We Summarize these Differences?}}: \blue{\textbf{Transfer learning}}}
\end{frame}

\begin{frame}{What is Transfer Learning?}
  \onslide<+->{\blue{\textbf{Def.}}: Taking knowledge learned from solving one problem and applying it to a different (related) one}
  \vfill
  \onslide<+->{Let's look at how each architecture uses transfer learning...}
  \begin{itemize}[<+->]
    \setlength{\itemsep}{20pt}
    \item \textbf{``Classic'' RNN}: \red{Limited}. \onslide<+->{Only (GloVe) embedding matrix}
    \item \textbf{ELMo Preprocessing}: \green{Extensive}. \onslide<+->{Entire ELMo network parameters static.  Only train a couple feed forward layers.}
  \end{itemize}
  \vfill
  \onslide<+->{\textbf{\green{Question}}: Is transfer learning a \textit{free lunch}?}

  \vspace{6pt}
  \onslide<+->{\textbf{Answer}: No -- there's never a free lunch.  \onslide<+->{Transfer learning is just an \textit{inductive bias} and like all biases limits some flexibility -- which may be good or bad.}}
\end{frame}

\begin{frame}{Experiments}
  Study Two Types of Bias
  \begin{itemize}
    \setlength{\itemsep}{10pt}
    \item Bias negative set covers only portion of ${\pDist(\X \vert \y = \pcls)}$'s support
  \end{itemize}

  \onslide<+->{We will hold off discussing the ``Classic'' RNN results until the end since it performs significantly worse}
\end{frame}

\begin{frame}{Architecture Comparison}

\end{frame}

\begin{frame}{Experiments -- Architectural Comparison}
  \begin{minipage}[t]{0.67\linewidth}
    \centering
    \vspace{-18pt}
    \onslide<+->{
      \begin{table}
        \caption{\small ELMo preprocessed accuracy results}
        \onslide<+->{
          {\footnotesize % Table atting based on: https://inf.ethz.ch/personal/markusp/teaching/guides/guide-tables.pdf
\begin{tabular}{@{}lllllllll@{}}
  \toprule
  \multicolumn{3}{c}{$\bntrain$ Bias} &    & \multicolumn{2}{c}{PN} &       &      \\\cmidrule{1-3}\cmidrule{5-6}
  sci.   & soc.   & talk.       &    & Unbiased                          & Biased          & nnPU                             & PUbN \\\midrule
  100\%  & 0\%    & 0\%         &    & \multicolumn{1}{c}{$\uparrow$}    & 0.766           & \multicolumn{1}{c}{$\uparrow$}   & \textbf{0.870}\\
  0\%    & 0\%    & 100\%       &    & \multicolumn{1}{c}{0.883}         & 0.814           & \multicolumn{1}{c}{0.834}        & \textbf{0.846} \\
  10\%   & 50\%   & 40\%        &    & \multicolumn{1}{c}{$\downarrow$}  & \textbf{0.872}  & \multicolumn{1}{c}{$\downarrow$} & 0.822 \\
  \bottomrule
\end{tabular}
}
        }
      \end{table}
    }
    \vspace{-20pt}
    \onslide<+->{
      \begin{table}
        \caption{\small ${\left(\text{Accuracy}_{ELMo} - \text{Accuracy}_{RNN}\right)}$\onslide<+->{: ${{>}0}$ means ELMO better}}
        \onslide<+->{
          {\footnotesize % Table atting based on: https://inf.ethz.ch/personal/markusp/teaching/guides/guide-tables.pdf
\begin{tabular}{@{}lllllllll@{}}
  \toprule
  \multicolumn{3}{c}{$\bntrain$ Bias} &    & \multicolumn{2}{c}{PN} &       &      \\\cmidrule{1-3}\cmidrule{5-6}
  sci.   & soc.   & talk.       &   & Unbiased                          & Biased          & nnPU                             & PUbN \\\midrule
  100\%  & 0\%    & 0\%         &   & \multicolumn{1}{c}{$\uparrow$}    & 0.126           & \multicolumn{1}{c}{$\uparrow$}   & 0.193 \\
  0\%    & 0\%    & 100\%       &   & \multicolumn{1}{c}{0.179}         & 0.046           & \multicolumn{1}{c}{0.230}        & 0.171 \\
  10\%   & 50\%   & 40\%        &   & \multicolumn{1}{c}{$\downarrow$}  & 0.111           & \multicolumn{1}{c}{$\downarrow$} & 0.221 \\
  \bottomrule
\end{tabular}
}
        }
      \end{table}
    }
    \vspace{-5pt}  % Need to fix false warning on frame size
  \end{minipage}
  \begin{minipage}{0.07\linewidth}
    \hspace{\fill}
  \end{minipage}
  \begin{minipage}[t]{0.23\linewidth}
    \vspace{40pt}
    \onslide<+->{%
      % \flushright
      \begin{block}{Major Takeaway}
        If training data limited/biased, transfer learning's benefits compound
      \end{block}
    }
  \end{minipage}
\end{frame}
\end{document}
