\documentclass[11pt,dvipsnames,usenames,aspectratio=169]{beamer}  % Add handout to options to disable overlays

% For more themes, color themes and font themes, see:
% http://deic.uab.es/~iblanes/beamer_gallery/index_by_theme.html
%
\mode<presentation>
{%
  \usetheme{CambridgeUS}    % or try default, Darmstadt, Warsaw, ...
  \usecolortheme{whale}     % or try albatross, beaver, crane, ...
  \usefonttheme{serif}          % or try default, structurebold, ...
  % \usefonttheme[onlymath]{serif}
  % \setbeamertemplate{navigation symbols}{}
  % \setbeamercovered{transparent}

  \setbeamercolor{title}{fg=white}
  \setbeamerfont{title}{series=\bfseries}
  \setbeamercolor{frametitle}{fg=black}
  \setbeamerfont{frametitle}{series=\bfseries}

  \setbeamercolor{section in head/foot}{fg=white}
  \setbeamerfont{section in head/foot}{series=\bfseries}
  \setbeamercolor{subsection in head/foot}{fg=white}
  \setbeamerfont{subsection in head/foot}{series=\bfseries}
  \setbeamercolor{author in head/foot}{fg=white}
  \setbeamerfont{author in head/foot}{series=\bfseries}
  \setbeamercolor{title in head/foot}{fg=white}
  \setbeamerfont{title in head/foot}{series=\bfseries}

  \setbeamercolor{block title}{use=structure,fg=white,bg=title in head/foot.bg}
  \setbeamerfont{block title}{series=\bfseries}
  \setbeamercolor{block body}{use=structure,fg=black,bg=black!3!white}
}

% Support graying out frame elements
\newcommand{\FrameOpague}{\setbeamercovered{again covered={\opaqueness<1->{40}}}}
% Transition slide
\newcommand{\transitionFrame}[1]{%
{%
  \begin{frame}[plain,noframenumbering]{}{} % the plain option removes the sidebar and header from the title page
    \setbeamertemplate{final page}[text]{\Large \textbf{#1}}
    \usebeamertemplate{final page}
  \end{frame}}
}

\newcommand{\toolname}{ddPU}

\newcommand{\Repel}{Repellent}
\newcommand{\repel}{\MakeLowercase{\Repel}}

\newcommand{\X}{x}
\newcommand{\domain}{\real^{d}}
\newcommand{\eX}[1]{\X_{(#1)}}
\newcommand{\xA}{\eX{A}}
\newcommand{\xP}{\eX{P}}
\newcommand{\xN}{\eX{N}}

\newcommand{\y}{y}
\newcommand{\labels}{\set{{\pm}1}}
\newcommand{\pCls}{{+}1}
\newcommand{\nCls}{{-}1}

\newcommand{\latent}{s}
\newcommand{\params}{\theta}

\newcommand{\pDist}{p}
\newcommand{\joint}{\pDist_{XY}}
\newcommand{\marginal}{\pDist_{X}}

\newcommand{\size}{n}
\newcommand{\train}{\mathcal{X}}
\newcommand{\trainVar}[1]{\train_{\textnormal{#1}}}
\newcommand{\unlabel}{\trainVar{U}}
\newcommand{\pos}{\trainVar{P}}
% Batch specific naming
\newcommand{\cnt}{i}
\newcommand{\unlI}[1]{#1{\mathcal{X}}_{\textnormal{U}}^{(\cnt)}}
\newcommand{\posI}[1]{#1{\mathcal{X}}_{\textnormal{P}}^{(\cnt)}}


% Siamese Network Related Macros
\newcommand{\margin}{\alpha}
\newcommand{\SiamDim}{m}
\newcommand{\SiamFunc}{f}
\newcommand{\distMetric}{\delta}
\newcommand{\siamDist}[3]{\norm{#1\left(#2\right) - #1\left(#3\right)}}
\newcommand{\lTrip}{\mathcal{L}_{\text{Triplet}}}

% Function names for the DeepPU architecture
\newcommand{\fPU}{g}
\newcommand{\fPUenc}{\fPU_{\text{enc}}}
\newcommand{\fPUp}{\fPU_{p}}
\newcommand{\fPUn}{\fPU_{n}}

\newcommand{\etal}{et~al.}
\newcommand{\elkan}{Elkan \&~Noto}

% Used for including standalone docs
\usepackage{standalone}

\newcommand{\transpose}{^{\text{T}}}

% Imported via UltiSnips
% Unbreakable dash:
%  Words hyphened with these dashes can also be broken at other positions than the dash
%    \-/ hyphen
%    \-- en-dash
%    \--- em-dash
%    extdash unbreakable dashes
%
%  No line breaks possible at the hyphen
%    \=/ hyphen
%    \== en-dash
%    \=== em-dash
\usepackage[shortcuts]{extdash}

% Imported via UltiSnips
\usepackage[dvipsnames]{xcolor}
\newcommand{\colortext}[2]{{\color{#1} #2}}
\newcommand{\red}[1]{\colortext{red}{#1}}
\newcommand{\blue}[1]{\colortext{blue}{#1}}
\newcommand{\green}[1]{\colortext{green}{#1}}

% Imported via UltiSnips
\usepackage{amsmath}
\DeclareMathOperator*{\argmax}{arg\,max}
\DeclareMathOperator*{\argmin}{arg\,min}
\usepackage{amsfonts}  % Used for \mathbb and \mathcal
\usepackage{amssymb}

% Imported via UltiSnips
\usepackage{mathtools} % for "\DeclarePairedDelimiter" macro
% \swapifbranches changes unstarred paired delimiters to starred and
% vice versa.  This means by default, paired delimiters have the star.
\usepackage{etoolbox}
\newcommand\swapifbranches[3]{#1{#3}{#2}}
\makeatletter
\MHInternalSyntaxOn
\patchcmd{\DeclarePairedDelimiter}{\@ifstar}{\swapifbranches\@ifstar}{}{}
\MHInternalSyntaxOff
\makeatother
% Place after swap to ensure swap star
\DeclarePairedDelimiter{\sbrack}{\lbrack}{\rbrack}
\DeclarePairedDelimiter{\floor}{\lfloor}{\rfloor}
\DeclarePairedDelimiter{\ceil}{\lceil}{\rceil}
\DeclarePairedDelimiter{\abs}{\lvert}{\rvert}
\DeclarePairedDelimiter{\norm}{\lVert}{\rVert}
\usepackage{bm}
\DeclarePairedDelimiterX\set[1]\lbrace\rbrace{#1}
\DeclarePairedDelimiterX\setbuild[2]\lbrace\rbrace{#1 \bm: #2}
\newcommand{\ints}[1]{{\sbrack{#1}}}
\newcommand{\func}[3]{{#1:#2\rightarrow#3}}
\newcommand{\defeq}{\stackrel{\mathclap{\mbox{\tiny def}}}{=}}

% Imported via UltiSnips
\usepackage{multirow}
\usepackage{booktabs}

% Imported via UltiSnips
\usepackage{tikz}
\usetikzlibrary{arrows,decorations.markings,shadows,positioning,calc,backgrounds,shapes}

% Imported via UltiSnips
\usepackage[noend]{algpseudocode}
\usepackage[Algorithm,ruled]{algorithm}
\algnewcommand\algorithmicforeach{\textbf{for each}}
\algdef{S}[FOR]{ForEach}[1]{\algorithmicforeach\ #1\ \algorithmicdo}
\newcommand{\algin}[1]{\hspace*{\algorithmicindent} \textbf{Input} #1\\}
\newcommand{\algout}[1]{\hspace*{\algorithmicindent} \textbf{Output} #1\\}


\usepackage{color}
\renewcommand{\blue}[1]{{\color{Blue} #1}}
\renewcommand{\red}[1]{{\color{red} #1}}
\renewcommand{\green}[1]{{\color{ForestGreen} #1}}

% Here's where the presentation starts, with the info for the title slide
\title[Negative Covariate Shift]{Address Negative Covariate Shift \\ on 20~Newsgroups Classification}
\author[Zayd Hammoudeh]{%
  \href{mailto:zayd@cs.uoregon.edu}{\textbf{Zayd Hammoudeh}}\inst{1}  % \textsuperscript{(\Letter)}
  % \and
  % \href{mailto:lowd@cs.uoregon.edu}{Daniel Lowd}\inst{1}
}

\institute[Univ.\ Oregon]{%
  \textsuperscript{1}\textbf{University of Oregon}\\
  Eugene, OR, USA\\
  \texttt{\href{mailto:zayd@cs.uoregon.edu}{zayd@cs.uoregon.edu}}
  % \texttt{{zayd, lowd}@ucsc.edu}
}
\date{\today}


\begin{document}

\begin{frame}
  \titlepage
\end{frame}

\begin{frame}{Problem Setting}
  \begin{itemize}
    \item Construct a supervised, binary text document classifier
  \end{itemize}
\end{frame}

\begin{frame}{Classifier Architectures}
  \onslide<+->{We considered two deep architectures:}
  \vfill
  \begin{itemize}[<+->]
    \setlength{\itemsep}{24pt}
    \item \blue{\textbf{``Classic'' RNN Classifier}}: Tokenize sentence, encode with embedding matrix, classify with RNN + feedforward

    \item \blue{\textbf{ELMo Preprocess}}: Encode entire document into a static representation with ELMo (a ``document embedding'').  Train a feedforward to classify the static vectors.
  \end{itemize}
  \vfill
  \onslide<+->{\green{\textbf{How Can We Summarize these Differences?}}: \blue{\textbf{Transfer learning}}}
\end{frame}

\begin{frame}{What is Transfer Learning?}
  \onslide<+->{\blue{\textbf{Def.}}: Taking knowledge learned from solving one problem and applying it to a different (related) one}
  \vfill
  \onslide<+->{Let's look at how each architecture uses transfer learning...}
  \begin{itemize}[<+->]
    \setlength{\itemsep}{20pt}
    \item \textbf{``Classic'' RNN}: \red{Limited}. \onslide<+->{Only (GloVe) embedding matrix}
    \item \textbf{ELMo Preprocessing}: \green{Extensive}. \onslide<+->{Entire ELMo network parameters static.  Only train a couple feed forward layers.}
  \end{itemize}
  \vfill
  \onslide<+->{\textbf{\green{Question}}: Is transfer learning a \textit{free lunch}?}

  \vspace{6pt}
  \onslide<+->{\textbf{Answer}: No -- there's never a free lunch.  \onslide<+->{Transfer learning is just an \textit{inductive bias} and like all biases limits some flexibility -- which may be good or bad.}}
\end{frame}

\begin{frame}{Experiments}
  Study Two Types of Bias
  \begin{itemize}
    \setlength{\itemsep}{10pt}
    \item Bias negative set covers only portion of ${\pDist(\X \vert \y = \pcls)}$'s support
  \end{itemize}

  \onslide<+->{We will hold off discussing the ``Classic'' RNN results until the end since it performs significantly worse}
\end{frame}

\begin{frame}{Architecture Comparison}

\end{frame}

\begin{frame}{Experiments -- Architectural Comparison}
  \begin{minipage}[t]{0.67\linewidth}
    \centering
    \vspace{-18pt}
    \onslide<+->{
      \begin{table}
        \caption{\small ELMo preprocessed accuracy results}
        \onslide<+->{
          {\footnotesize % Table atting based on: https://inf.ethz.ch/personal/markusp/teaching/guides/guide-tables.pdf
\begin{tabular}{@{}lllllllll@{}}
  \toprule
  \multicolumn{3}{c}{$\bntrain$ Bias} &    & \multicolumn{2}{c}{PN} &       &      \\\cmidrule{1-3}\cmidrule{5-6}
  sci.   & soc.   & talk.       & $\rho$   & Unbiased                          & Biased          & nnPU                             & PUbN \\\midrule
  100\%  & 0\%    & 0\%         & 0.21     & \multicolumn{1}{c}{$\uparrow$}    & 0.766           & \multicolumn{1}{c}{$\uparrow$}   & \textbf{0.870}\\
  0\%    & 0\%    & 100\%       & 0.17     & \multicolumn{1}{c}{0.883}         & 0.814           & \multicolumn{1}{c}{0.834}        & \textbf{0.846} \\
  10\%   & 50\%   & 40\%        & 0.1      & \multicolumn{1}{c}{$\downarrow$}  & \textbf{0.872}  & \multicolumn{1}{c}{$\downarrow$} & 0.822 \\
  \bottomrule
\end{tabular}
}
        }
      \end{table}
    }
    \vspace{-20pt}
    \onslide<+->{
      \begin{table}
        \caption{\small ${\left(\text{Accuracy}_{ELMo} - \text{Accuracy}_{RNN}\right)}$\onslide<+->{: ${{>}0}$ means ELMO better}}
        \onslide<+->{
          {\footnotesize % Table atting based on: https://inf.ethz.ch/personal/markusp/teaching/guides/guide-tables.pdf
\begin{tabular}{@{}lllllllll@{}}
  \toprule
  \multicolumn{3}{c}{$\bntrain$ Bias} &    & \multicolumn{2}{c}{PN} &       &      \\\cmidrule{1-3}\cmidrule{5-6}
  sci.   & soc.   & talk.       &   & Unbiased                          & Biased          & nnPU                             & PUbN \\\midrule
  100\%  & 0\%    & 0\%         &   & \multicolumn{1}{c}{$\uparrow$}    & 0.126           & \multicolumn{1}{c}{$\uparrow$}   & 0.193 \\
  0\%    & 0\%    & 100\%       &   & \multicolumn{1}{c}{0.179}         & 0.046           & \multicolumn{1}{c}{0.230}        & 0.171 \\
  10\%   & 50\%   & 40\%        &   & \multicolumn{1}{c}{$\downarrow$}  & 0.111           & \multicolumn{1}{c}{$\downarrow$} & 0.221 \\
  \bottomrule
\end{tabular}
}
        }
      \end{table}
    }
    \vspace{-5pt}  % Need to fix false warning on frame size
  \end{minipage}
  \begin{minipage}{0.07\linewidth}
    \hspace{\fill}
  \end{minipage}
  \begin{minipage}[t]{0.23\linewidth}
    \vspace{40pt}
    \onslide<+->{%
      % \flushright
      \begin{block}{Major Takeaway}
        If training data limited/biased, transfer learning's benefits compound
      \end{block}
    }
  \end{minipage}
\end{frame}
\end{document}
