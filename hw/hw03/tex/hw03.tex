\documentclass{article}

\newcommand{\name}{Zayd Hammoudeh}
\newcommand{\course}{CIS510: Natural Lang. Processing}
\newcommand{\assnName}{HW\#3}
\newcommand{\dueDate}{November~16, 2019}

\usepackage[margin=1in]{geometry}
\usepackage[skip=4pt]{caption}      % ``skip'' sets the spacing between the figure and the caption.
\usepackage{tikz}
\usetikzlibrary{arrows.meta,decorations.markings,shadows,positioning,calc}
\usepackage{pgfplots}               % Needed for plotting
\pgfplotsset{compat=newest}
\usepgfplotslibrary{fillbetween}    % Allow for highlighting under a curve
\usepackage{amsmath}                % Allows for piecewise functions using the ``cases'' construct
\usepackage{siunitx}                % Allows for ``S'' alignment in table to align by decimal point

\usepackage[obeyspaces,spaces]{url} % Used for typesetting with the ``path'' command
\usepackage[hidelinks]{hyperref}    % Make the cross references clickable hyperlinks
\usepackage[bottom]{footmisc}       % Prevents the table going below the footnote
\usepackage{nccmath}                % Needed in the workaround for the ``aligncustom'' environment
\usepackage{amssymb}                % Used for black QED symbol
\usepackage{bm}                     % Allows for bolding math symbols.
\usepackage{tabto}                  % Allows to tab to certain point on a line
\usepackage{float}
\usepackage{subcaption}             % Allows use of the ``subfigure'' environment
\usepackage{enumerate}              % Allow enumeration other than just numbers

\usepackage[noend]{algpseudocode}
\usepackage[Algorithm,ruled]{algorithm}
\algnewcommand\algorithmicforeach{\textbf{for each}}
\algdef{S}[FOR]{ForEach}[1]{\algorithmicforeach\ #1\ \algorithmicdo}

%---------------------------------------------------%
%     Define Distances Used for Document Margins    %
%---------------------------------------------------%

\newcommand{\hangindentdistance}{1cm}
\newcommand{\defaultleftmargin}{0.25in}
\newcommand{\questionleftmargin}{-.5in}

\setlength{\parskip}{1em}
\setlength{\oddsidemargin}{\defaultleftmargin}

%---------------------------------------------------%
%      Configure the Document Header and Footer     %
%---------------------------------------------------%

% Set up page formatting
\usepackage{todonotes}
\usepackage{fancyhdr}                   % Used for every page footer and title.
\pagestyle{fancy}
\fancyhf{}                              % Clears both the header and footer
\renewcommand{\headrulewidth}{0pt}      % Eliminates line at the top of the page.
\fancyfoot[LO]{\course\ -- \assnName}   % Left
\fancyfoot[CO]{\thepage}                % Center
\fancyfoot[RO]{\name}                   % Right

%---------------------------------------------------%
%           Define the Title Page Entries           %
%---------------------------------------------------%

\title{\textbf{\course\ -- \assnName}}
\author{\name}

%---------------------------------------------------%
% Define the Environments for the Problem Inclusion %
%---------------------------------------------------%

\usepackage{scrextend}
\newcounter{problemCount}
\setcounter{problemCount}{0}  % Reset the subproblem counter

\newcounter{subProbCount}[problemCount]   % Reset subProbCount any time problemCount changes.
\renewcommand{\thesubProbCount}{\alph{subProbCount}}  % Make it so the counter is referenced as a number

\newenvironment{problemshell}{
  \begin{addmargin}[\questionleftmargin]{0em}
    \par%
    \medskip
    \leftskip=0pt\rightskip=0pt%
    \setlength{\parindent}{0pt}
    \bfseries
  }
  {
    \par\medskip
  \end{addmargin}
}
\newenvironment{problem}
{%
  \refstepcounter{problemCount} % Increment the subproblem counter.  This must be before the exercise to ensure proper numbering of claims and lemmas.
  \begin{problemshell}
    \noindent \textit{Exercise~\#\arabic{problemCount}} \\
  }
  {
  \end{problemshell}
  %  \setcounter{subProbCount}{0} % Reset the subproblem counter
}
\newenvironment{subproblem}
{%
  \begin{problemshell}
    \refstepcounter{subProbCount} % Increment the subproblem counter
    \setlength{\leftskip}{\hangindentdistance}
    % Print the subproblem count and offset to the left
    \hspace{-\hangindentdistance}(\alph{subProbCount}) \tabto{0pt}
  }
  {
  \end{problemshell}
}

% Change interline spacing.
\renewcommand{\baselinestretch}{1.1}
\newenvironment{aligncustom}
{ \csname align*\endcsname % Need to do this instead of \begin{align*} because of LaTeX bug.
  \centering
}
{
  \csname endalign*\endcsname
}

%---------------------------------------------------%
%       Define commands related to managing         %
%    floats (e.g., images) across multiple pages    %
%---------------------------------------------------%

\usepackage{placeins}     % Allows \FloatBarrier

% Prevent preceding floats going to this page
\newcommand{\floatnewpage}{\FloatBarrier\newpage}

% Add the specified input file and prevent any floated figures/tables going onto the same page as new input
\newcommand{\problemFile}[1]{
  \floatnewpage
  \input{#1}
}

\newcommand{\etal}{et~al.}
\newcommand{\elkan}{Elkan \&~Noto}

% Used for including standalone docs
\usepackage{standalone}

\newcommand{\transpose}{^{\text{T}}}

% Imported via UltiSnips
% Unbreakable dash:
%  Words hyphened with these dashes can also be broken at other positions than the dash
%    \-/ hyphen
%    \-- en-dash
%    \--- em-dash
%    extdash unbreakable dashes
%
%  No line breaks possible at the hyphen
%    \=/ hyphen
%    \== en-dash
%    \=== em-dash
\usepackage[shortcuts]{extdash}

% Imported via UltiSnips
\usepackage[dvipsnames]{xcolor}
\newcommand{\colortext}[2]{{\color{#1} #2}}
\newcommand{\red}[1]{\colortext{red}{#1}}
\newcommand{\blue}[1]{\colortext{blue}{#1}}
\newcommand{\green}[1]{\colortext{green}{#1}}

% Imported via UltiSnips
\usepackage{amsmath}
\DeclareMathOperator*{\argmax}{arg\,max}
\DeclareMathOperator*{\argmin}{arg\,min}
\usepackage{amsfonts}  % Used for \mathbb and \mathcal
\usepackage{amssymb}

% Imported via UltiSnips
\usepackage{mathtools} % for "\DeclarePairedDelimiter" macro
% \swapifbranches changes unstarred paired delimiters to starred and
% vice versa.  This means by default, paired delimiters have the star.
\usepackage{etoolbox}
\newcommand\swapifbranches[3]{#1{#3}{#2}}
\makeatletter
\MHInternalSyntaxOn
\patchcmd{\DeclarePairedDelimiter}{\@ifstar}{\swapifbranches\@ifstar}{}{}
\MHInternalSyntaxOff
\makeatother
% Place after swap to ensure swap star
\DeclarePairedDelimiter{\sbrack}{\lbrack}{\rbrack}
\DeclarePairedDelimiter{\floor}{\lfloor}{\rfloor}
\DeclarePairedDelimiter{\ceil}{\lceil}{\rceil}
\DeclarePairedDelimiter{\abs}{\lvert}{\rvert}
\DeclarePairedDelimiter{\norm}{\lVert}{\rVert}
\usepackage{bm}
\DeclarePairedDelimiterX\set[1]\lbrace\rbrace{#1}
\DeclarePairedDelimiterX\setbuild[2]\lbrace\rbrace{#1 \bm: #2}
\newcommand{\ints}[1]{{\sbrack{#1}}}
\newcommand{\func}[3]{{#1:#2\rightarrow#3}}
\newcommand{\defeq}{\stackrel{\mathclap{\mbox{\tiny def}}}{=}}

% Imported via UltiSnips
\usepackage{multirow}
\usepackage{booktabs}

% Imported via UltiSnips
\usepackage{tikz}
\usetikzlibrary{arrows,decorations.markings,shadows,positioning,calc,backgrounds,shapes}

% Imported via UltiSnips
\usepackage[noend]{algpseudocode}
\usepackage[Algorithm,ruled]{algorithm}
\algnewcommand\algorithmicforeach{\textbf{for each}}
\algdef{S}[FOR]{ForEach}[1]{\algorithmicforeach\ #1\ \algorithmicdo}
\newcommand{\algin}[1]{\hspace*{\algorithmicindent} \textbf{Input} #1\\}
\newcommand{\algout}[1]{\hspace*{\algorithmicindent} \textbf{Output} #1\\}

\newcommand{\points}[1]{\textnormal{(#1~Points)}}


\begin{document}
  % \maketitle

  \noindent
  \textbf{Name}: \name\\
  \textbf{Course}: \course\\
  \textbf{Assignment}: \assnName\\
  \textbf{Due Date}: \dueDate

  % \noindent
  % \textbf{Other Student Discussions}: I discussed the problems in this homework with the following student(s) below.  All write-ups were prepared independently.
  % \vspace{-1em}
  % \begin{itemize}
  %   \item Viet Lai
  %   \item Dave Patterson
  % \end{itemize}

  \section{Development Set Performance Metrics}

  \begin{itemize}
    \setlength{\itemsep}{0pt}
    \item \textbf{Accuracy}: \blue{96.45\%}
    \item \textbf{Precision}: \blue{0.7645}
    \item \textbf{Recall}: \blue{0.7823}
    \item \textbf{F1-Score}: \blue{0.7733}
  \end{itemize}

  \section{Implementation Notes}

  This section describes the features I tried when creating my learner.  Those features which degraded performance and were removed are noted.

  \begin{itemize}
    \item \textbf{Capitalization}:
      \begin{itemize}
        \item \textit{has\_cap}: Binary feature that equals \texttt{0b1} if the token word has any capital letter.
        \item \textit{prev\_has\_cap}: Binary feature that equals \texttt{0b1} if the preceding token (if any) has any capital letter. This feature degraded performance and was removed.
        \item \textit{next\_has\_cap}: Binary feature that equals \texttt{0b1} if the succeeding token (if any) has any capital letter. This feature degraded performance and was removed.
        \item \textit{all\_caps}: Binary feature that equals \texttt{0b1} if the token word is all capital letters.
      \end{itemize}
    \item \textbf{Chunks}:
      \begin{itemize}
        \item \textit{prev\_chunk\_mismatch}: Binary feature that equals \texttt{0b1} if the current token's chunk is different from the \textit{previous} token's chunk.
        \item \textit{next\_chunk\_mismatch}: Binary feature that equals \texttt{0b1} if the current token's chunk is different from the \textit{next} token's chunk.
        \item \textit{is\_np}: Binary feature that equals \texttt{0b1} if the current token's chunk is a noun phrase.
      \end{itemize}
    \item \textbf{Miscellaneous}:
      \begin{itemize}
        \item \textit{any\_punc}: Binary feature that equals \texttt{0b1} if any punctuation symbol appears in the current token.  This feature either had no effect or slightly degraded performance so it was removed.
        \item \textit{is\_first}: Binary feature that equals \texttt{0b1} if the token is the first word in the sequence.
        \item \textit{idx}: Index of the word in the sentence.  This feature decreased accuracy and was removed.
      \end{itemize}
    \item \textbf{Parts of Speech}
      \begin{itemize}
        \item \textit{is\_adj}: Binary feature that equals \texttt{0b1} if the token word is any form of adjective.
        \item \textit{is\_dig}: Binary feature that equals \texttt{0b1} if the token word is a digit.  This feature did not improve performance and was removed.
        \item \textit{is\_fw}: Binary feature that equals \texttt{0b1} if the token is a foreign word.  This feature either had no effect or slightly degraded performance so it was removed.
        \item \textit{is\_noun}: Binary feature that equals \texttt{0b1} if the token word is any form of noun.
        \item \textit{is\_punc}: Binary feature that equals \texttt{0b1} if the token word is any form of punctuation.
        \item \textit{is\_sym}: Binary feature that equals \texttt{0b1} if the token word is a symbolic part of speech.  This feature either had no effect or slightly degraded performance so it was removed.
        \item \textit{is\_verb}: Binary feature that equals \texttt{0b1} if the token word is any form of verb.
      \end{itemize}
  \end{itemize}

  \subsection{More Advanced Features}

  \begin{itemize}
    \item \textbf{Brown Clustering}: Using the file supplied with the homework, Brown bit strings of length 2, 4, 6, 8, 10, and~11 bits are included as features.  This approach improved accuracy by approximately 0.75\%.
    \item \texttt{LargestCity.txt}: This file was supplied as part of the homework, and my program has three features associated with this city list.  They are:
      \begin{itemize}
        \item \texttt{Token Word}: Binary feature which is \texttt{0b1} if the token word itself appears in the city list.
        \item \texttt{Previous Token} + \texttt{Token Word}: Binary feature which equals \texttt{0b1} if the previous token word (if any) and the current token word appear together in the city list as a pair.
        \item \texttt{Token Word} + \texttt{Next Token}: Binary feature which equals \texttt{0b1} if the current token word and the next token word (if any) appear together in the city list as a pair.
      \end{itemize}
    \item \textbf{Major Cities of the World Dataset}: Sourced from \href{https://datahub.io/core/world-cities}{datahub.io} based on the \href{http://www.geonames.org/}{GeoNames database}.  Using this information, fields were also added for state/region and country information for major cities around the world.
    \item \texttt{names}: Python package with common first and last names.  My program has three features associated with the names set.  They are:
      \begin{itemize}
        \item \texttt{Female First Name}: If the token word matches a popular \textit{female} first name, a real-valued feature is added corresponding to the percentage of females with that name.  Otherwise, this feature equals zero.
        \item \texttt{Male First Name}: If the token word matches a popular \textit{male} first name, a real-valued feature is added corresponding to the percentage of males with that name.  Otherwise, this feature equals zero.
        \item \texttt{Last Name}: If the token word matches a popular \textit{last} name, a real-valued feature is added corresponding to the percentage of people with that last name.  Otherwise, this feature equals zero.
      \end{itemize}
    \item \texttt{names-dataset}: Collection of \textasciitilde160,000~first names and \textasciitilde100,000 last names in a Python package.  Binary fields were added for whether the current, previous, and next tokens are in this gazetteer dataset.

  \end{itemize}

  \section{Running the Program}

  The implementation was tested using Python~3.7.1.  The script is evoked on the command line as shown below.  Also included is a description of each of the command parameters.

  \begin{center}
    \texttt{python max\_ent\_trainer.py train test out [--key key\_file]}
  \end{center}

  \begin{itemize}
    \item \texttt{train}: Path to the \texttt{*.pos-chunk-name} training file
    \item \texttt{test}: Path to the (test) \texttt{*.pos-chunk} file to be labeled
    \item \texttt{out}: Path to the output file to write
    \item \texttt{key\_file}: Optionally allows the user to specify a key file.  If specified, the scorer script is automatically run comparing \texttt{key\_file} and \texttt{out}.
  \end{itemize}

  \subsection{Non-standard Python Packages}

  The following non-standard packages are used in this program:

  \begin{itemize}
    \item \texttt{names}: This contains common US names based on the 1990~census. Any name in the database is associated with a popularity score making it more insightful than just whether a word is a name.
    \item \texttt{names-dataset}: A large dataset of 160,000 first names and 100,000 last names.
    \item \texttt{string}: Python package that contains a set of all punctuation symbols.
    \item \texttt{subprocess}: Used to manage the automatic building and running of the Java code.
  \end{itemize}
\end{document}
