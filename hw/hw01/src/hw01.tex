\documentclass{report}

\newcommand{\name}{Zayd Hammoudeh}
\newcommand{\course}{CIS510: Natural Lang. Procesisng}
\newcommand{\assnName}{HW\#1}
\newcommand{\dueDate}{October~22, 2019}

\usepackage[margin=1in]{geometry}
\usepackage[skip=4pt]{caption}      % ``skip'' sets the spacing between the figure and the caption.
\usepackage{tikz}
\usetikzlibrary{arrows.meta,decorations.markings,shadows,positioning,calc}
\usepackage{pgfplots}               % Needed for plotting
\pgfplotsset{compat=newest}
\usepgfplotslibrary{fillbetween}    % Allow for highlighting under a curve
\usepackage{amsmath}                % Allows for piecewise functions using the ``cases'' construct
\usepackage{siunitx}                % Allows for ``S'' alignment in table to align by decimal point

\usepackage[obeyspaces,spaces]{url} % Used for typesetting with the ``path'' command
\usepackage[hidelinks]{hyperref}    % Make the cross references clickable hyperlinks
\usepackage[bottom]{footmisc}       % Prevents the table going below the footnote
\usepackage{nccmath}                % Needed in the workaround for the ``aligncustom'' environment
\usepackage{amssymb}                % Used for black QED symbol
\usepackage{bm}                     % Allows for bolding math symbols.
\usepackage{tabto}                  % Allows to tab to certain point on a line
\usepackage{float}
\usepackage{subcaption}             % Allows use of the ``subfigure'' environment
\usepackage{enumerate}              % Allow enumeration other than just numbers

\usepackage[noend]{algpseudocode}
\usepackage[Algorithm,ruled]{algorithm}
\algnewcommand\algorithmicforeach{\textbf{for each}}
\algdef{S}[FOR]{ForEach}[1]{\algorithmicforeach\ #1\ \algorithmicdo}

%---------------------------------------------------%
%     Define Distances Used for Document Margins    %
%---------------------------------------------------%

\newcommand{\hangindentdistance}{1cm}
\newcommand{\defaultleftmargin}{0.25in}
\newcommand{\questionleftmargin}{-.5in}

\setlength{\parskip}{1em}
\setlength{\oddsidemargin}{\defaultleftmargin}

%---------------------------------------------------%
%      Configure the Document Header and Footer     %
%---------------------------------------------------%

% Set up page formatting
\usepackage{todonotes}
\usepackage{fancyhdr}                   % Used for every page footer and title.
\pagestyle{fancy}
\fancyhf{}                              % Clears both the header and footer
\renewcommand{\headrulewidth}{0pt}      % Eliminates line at the top of the page.
\fancyfoot[LO]{\course\ -- \assnName}   % Left
\fancyfoot[CO]{\thepage}                % Center
\fancyfoot[RO]{\name}                   % Right

%---------------------------------------------------%
%           Define the Title Page Entries           %
%---------------------------------------------------%

\title{\textbf{\course\ -- \assnName}}
\author{\name}

%---------------------------------------------------%
% Define the Environments for the Problem Inclusion %
%---------------------------------------------------%

\usepackage{scrextend}
\newcounter{problemCount}
\setcounter{problemCount}{0}  % Reset the subproblem counter

\newcounter{subProbCount}[problemCount]   % Reset subProbCount any time problemCount changes.
\renewcommand{\thesubProbCount}{\alph{subProbCount}}  % Make it so the counter is referenced as a number

\newenvironment{problemshell}{
  \begin{addmargin}[\questionleftmargin]{0em}
    \par%
    \medskip
    \leftskip=0pt\rightskip=0pt%
    \setlength{\parindent}{0pt}
    \bfseries
  }
  {
    \par\medskip
  \end{addmargin}
}
\newenvironment{problem}
{%
  \refstepcounter{problemCount} % Increment the subproblem counter.  This must be before the exercise to ensure proper numbering of claims and lemmas.
  \begin{problemshell}
    \noindent \textit{Exercise~\#\arabic{problemCount}} \\
  }
  {
  \end{problemshell}
  %  \setcounter{subProbCount}{0} % Reset the subproblem counter
}
\newenvironment{subproblem}
{%
  \begin{problemshell}
    \refstepcounter{subProbCount} % Increment the subproblem counter
    \setlength{\leftskip}{\hangindentdistance}
    % Print the subproblem count and offset to the left
    \hspace{-\hangindentdistance}(\alph{subProbCount}) \tabto{0pt}
  }
  {
  \end{problemshell}
}

% Change interline spacing.
\renewcommand{\baselinestretch}{1.1}
\newenvironment{aligncustom}
{ \csname align*\endcsname % Need to do this instead of \begin{align*} because of LaTeX bug.
  \centering
}
{
  \csname endalign*\endcsname
}

%---------------------------------------------------%
%       Define commands related to managing         %
%    floats (e.g., images) across multiple pages    %
%---------------------------------------------------%

\usepackage{placeins}     % Allows \FloatBarrier

% Prevent preceding floats going to this page
\newcommand{\floatnewpage}{\FloatBarrier\newpage}

% Add the specified input file and prevent any floated figures/tables going onto the same page as new input
\newcommand{\problemFile}[1]{
  \floatnewpage
  \input{#1}
}

\newcommand{\etal}{et~al.}
\newcommand{\elkan}{Elkan \&~Noto}

% Used for including standalone docs
\usepackage{standalone}

\newcommand{\transpose}{^{\text{T}}}

% Imported via UltiSnips
% Unbreakable dash:
%  Words hyphened with these dashes can also be broken at other positions than the dash
%    \-/ hyphen
%    \-- en-dash
%    \--- em-dash
%    extdash unbreakable dashes
%
%  No line breaks possible at the hyphen
%    \=/ hyphen
%    \== en-dash
%    \=== em-dash
\usepackage[shortcuts]{extdash}

% Imported via UltiSnips
\usepackage[dvipsnames]{xcolor}
\newcommand{\colortext}[2]{{\color{#1} #2}}
\newcommand{\red}[1]{\colortext{red}{#1}}
\newcommand{\blue}[1]{\colortext{blue}{#1}}
\newcommand{\green}[1]{\colortext{green}{#1}}

% Imported via UltiSnips
\usepackage{amsmath}
\DeclareMathOperator*{\argmax}{arg\,max}
\DeclareMathOperator*{\argmin}{arg\,min}
\usepackage{amsfonts}  % Used for \mathbb and \mathcal
\usepackage{amssymb}

% Imported via UltiSnips
\usepackage{mathtools} % for "\DeclarePairedDelimiter" macro
% \swapifbranches changes unstarred paired delimiters to starred and
% vice versa.  This means by default, paired delimiters have the star.
\usepackage{etoolbox}
\newcommand\swapifbranches[3]{#1{#3}{#2}}
\makeatletter
\MHInternalSyntaxOn
\patchcmd{\DeclarePairedDelimiter}{\@ifstar}{\swapifbranches\@ifstar}{}{}
\MHInternalSyntaxOff
\makeatother
% Place after swap to ensure swap star
\DeclarePairedDelimiter{\sbrack}{\lbrack}{\rbrack}
\DeclarePairedDelimiter{\floor}{\lfloor}{\rfloor}
\DeclarePairedDelimiter{\ceil}{\lceil}{\rceil}
\DeclarePairedDelimiter{\abs}{\lvert}{\rvert}
\DeclarePairedDelimiter{\norm}{\lVert}{\rVert}
\usepackage{bm}
\DeclarePairedDelimiterX\set[1]\lbrace\rbrace{#1}
\DeclarePairedDelimiterX\setbuild[2]\lbrace\rbrace{#1 \bm: #2}
\newcommand{\ints}[1]{{\sbrack{#1}}}
\newcommand{\func}[3]{{#1:#2\rightarrow#3}}
\newcommand{\defeq}{\stackrel{\mathclap{\mbox{\tiny def}}}{=}}

% Imported via UltiSnips
\usepackage{multirow}
\usepackage{booktabs}

% Imported via UltiSnips
\usepackage{tikz}
\usetikzlibrary{arrows,decorations.markings,shadows,positioning,calc,backgrounds,shapes}

% Imported via UltiSnips
\usepackage[noend]{algpseudocode}
\usepackage[Algorithm,ruled]{algorithm}
\algnewcommand\algorithmicforeach{\textbf{for each}}
\algdef{S}[FOR]{ForEach}[1]{\algorithmicforeach\ #1\ \algorithmicdo}
\newcommand{\algin}[1]{\hspace*{\algorithmicindent} \textbf{Input} #1\\}
\newcommand{\algout}[1]{\hspace*{\algorithmicindent} \textbf{Output} #1\\}

\newcommand{\points}[1]{\textnormal{(#1~Points)}}


\begin{document}
  \maketitle

  \noindent
  \textbf{Name}: \name\\
  \textbf{Course}: \course\\
  \textbf{Assignment}: \assnName\\
  \textbf{Due Date}: \dueDate

  % \noindent
  % \textbf{Other Student Discussions}: I discussed the problems in this homework with the following student(s) below.  All write-ups were prepared independently.
  % \vspace{-1em}
  % \begin{itemize}
  %   \item <Enter Other Student Name>
  % \end{itemize}

  \begin{problem}
  \points{40} [\textnormal{Document Similarity}] Suppose our pets have produced two documents:

  D1 = [\texttt{woof woof meow}] \\ D2 = [\texttt{woof woof squeak}]
\end{problem}

\begin{subproblem}\label{P01:A}
  \points{15} What is the cosine similarity of D1 and D2, not using idf weighting?
\end{subproblem}

  \begin{table}[h]
    \centering
    \caption{Problem~\ref{P01:A} Term Frequencies}\label{tab:P01:TermFreq}
    \begin{tabular}{|c|c|c|c|}
      \hline
      \textbf{Document} & \texttt{woof} & \texttt{meow} & \texttt{squeak} \\\hline\hline
      D1                & 2             & 1             & 0 \\\hline
      D2                & 2             & 0             & 1 \\\hline
    \end{tabular}
  \end{table}

The cosine similar is defined in Eq.~\eqref{eq:P01:Cosine}.

\begin{equation}\label{eq:P01:Cosine}
  \text{similarity} = \frac{D1 \cdot D2}{\norm{D1}_{2}\norm{D2}_{2}}
\end{equation}

\noindent
Therefore, given the term frequencies in Table~\ref{tab:P01:TermFreq}, the cosine similarity is:

\begin{aligncustom}
  \text{similarity} &= \frac{2 * 2 + 1 * 0 + 0 * 1}{\sqrt{2^2 + 1^2 + 0^2} * \sqrt{2^2 + 0^2 + 1^2}} \\
                    &= \frac{4}{\sqrt{5}} = \boxed{0.8} \text{.}
\end{aligncustom}

\begin{subproblem}\label{P01:B}
  \points{15} What is the cosine similarity if idf weighting is used?
\end{subproblem}

The inverse document frequency,~$idf_i$, for the $i^{\text{th}}$ word is defined in Eq.~\eqref{eq:P01:IDF}, where $N$ is the number of document (e.g.,~size of the corpus) while $n_i$~is the number of documents containing the $i^{\text{th}}$~term.

\begin{equation}\label{eq:P01:IDF}
  idf = \log \left( \frac{N}{n_i} \right)
\end{equation}

\noindent
The weighting for each of the words in D1 and~D2 is shown in Table~\ref{tab:P01:IDF}.

  \begin{table}[h]
    \centering
    \caption{Problem~\ref{P01:B} Inverse Document Frequencies}\label{tab:P01:IDF}
    \begin{tabular}{|c|c|c|c|}
      \hline
          & \texttt{woof} & \texttt{meow}  & \texttt{squeak} \\\hline\hline
      IDF & 0             & $\log 2$       & $\log 2$ \\\hline
    \end{tabular}
  \end{table}

\noindent
The TF-IDF vectors are shown in Table~\ref{tab:P01:TFIDF}; the two vectors are orthogonal so their cosine similarity is~\boxed{0}.

  \begin{table}[h]
    \centering
    \caption{Problem~\ref{P01:A} Term Frequencies}\label{tab:P01:TFIDF}
    \begin{tabular}{|c|c|c|c|}
      \hline
      \textbf{Document} & \texttt{woof} & \texttt{meow} & \texttt{squeak} \\\hline\hline
      D1                & 0             & $\log 2$      & 0 \\\hline
      D2                & 0             & 0             & $\log 2$ \\\hline
    \end{tabular}
  \end{table}

\begin{subproblem}\label{P01:C}
  \points{10} How would the answer to~(\ref{P01:B}) change if we added a third document:

  D3 = [\texttt{meow squeak}]

  \noindent
  to the collection? [10 points]
\end{subproblem}

D3~changes the IDF such that each word appears in two of the three documents.  Therefore, IDF can be treated as a constant scalar that cancels out when used in the similar calculations.  This means the cosine similarity is the same as in Problem~\ref{P01:A} and equals~$\boxed{0.8}$.

  \begin{problem}
  \points{40} \textnormal{[Naive Bayes and smoothing]} Do exercises 4.1 and 4.2 in third (on-line) edition of the text (\url{https://web.stanford.edu/~jurafsky/slp3/ed3book.pdf}) (page 81). Show the intermediate steps in your calculation. Compute using probabilities, not logs of probabilities (so instead of adding logs of probabilities, you multiply probabilities).
\end{problem}

\begin{subproblem}\label{P02:A}
  Assume the following likelihoods for each word being part of a positive or negative movie review, and equal prior probabilities for each class.

  \begin{table}[h]
    \centering
    \caption{Jurafsky Exercise~4.1 word probabilities}\label{tab:}
    \begin{tabular}{ccc}
              & pos  & neg \\\hline
      I       & 0.09 & 0.16 \\
      always  & 0.07 & 0.06 \\
      like    & 0.29 & 0.06 \\
      foreign & 0.04 & 0.15 \\
      films   & 0.08 & 0.11 \\
    \end{tabular}
  \end{table}

  What  class  will  Naive  bayes  assign  to  the  sentence  ``I  always  like  foreign films.''
\end{subproblem}

For a class~${c \in \mathcal{C}}$, the Naive Bayes probability is:

\begin{aligncustom}
  \Pr \sbrack{c \vert \vec{w}} &= \frac{\Pr\sbrack{c} \Pr\sbrack{\vec{w} \vert c}}{\Pr\sbrack{\vec{w}}} \\
                               &= \frac{\Pr\sbrack{c} \prod_{w \in \vec{w}} \Pr\sbrack{w \vert c}}{\Pr\sbrack{\vec{w}}}\text{.}
\end{aligncustom}

\noindent
When doing the analysis in this problem, ${\Pr\sbrack{\vec{w}}}$~applies for all ${c \in \mathcal{C}}$ so it can be ignored in the calculations.  Similarly, since ${\Pr\sbrack{\text{pos}} = \Pr\sbrack{\text{neg}}}$, ${\Pr\sbrack{c}}$ can also be ignored.  Therefore:

\begin{aligncustom}
  \Pr\sbrack{\text{pos} \vert \vec{w}} &\propto 0.09 * 0.07 * 0.29 * 0.04 * 0.08 \\
                                       &\approx \boxed{5.85 * 10^{-6}}\text{.}
\end{aligncustom}

\begin{aligncustom}
  \Pr\sbrack{\text{neg} \vert \vec{w}} &\propto 0.16 * 0.06 * 0.06 * 0.15 * 0.11 \\
                                       &\approx \boxed{9.50 * 10^{-6}}\text{.}
\end{aligncustom}


\noindent
Therefore, it is more likely that the review ``I always like foreign films'' is \boxed{\text{neg}}.

\newpage
\begin{subproblem}\label{P02:B}
Given the following short movie reviews,  each labeled with a genre,  either comedy or action:

\begin{enumerate}
  \item fun, couple, love, love \hspace{6pt}\textbf{comedy}
  \item fast, furious, shoot \hspace{6pt}\textbf{action}
  \item couple, fly, fast, fun, fun \hspace{6pt}\textbf{comedy}
  \item furious, shoot, shoot, fun \hspace{6pt}\textbf{action}
  \item fly, fast, shoot, love \hspace{6pt}\textbf{action}
\end{enumerate}

\noindent
and a new document~D:

\noindent
\hspace{6pt}
fast, couple, shoot, fly

\noindent
compute the most likely class for~D. Assume a naive Bayes classifier and use add\-/1 smoothing for the likelihoods.
\end{subproblem}

From the above description, there are three \textbf{action} and two \textbf{comedy} movies.  Therefore,\linebreak ${\Pr\sbrack{\text{action}}=\frac{3}{5}=0.6}$ and ${\Pr\sbrack{\text{comedy}}=\frac{2}{5}=0.4}$.  The unnormalized word probabilities are shown in Table~\ref{tab:P02:UnnormalWordProb}.  The Bernoulli model for word probability was used where:

\begin{equation}\label{eq:P02:BernoulliWordProb}
  \Pr\sbrack{w_i \vert c} = \frac{\text{\# Docs Labeled } c \text{ containing } w_i}{\text{\# Docs Labeled } c}
\end{equation}

\noindent
Observe that only the words relevant to this problem (e.g.,~fast, couple, shoot, fly) are shown in the table.

\begin{table}[h]
  \centering
  \caption{Jurafsky Exercise~4.2 unnormalized word probabilities}\label{tab:P02:UnnormalWordProb}
  \begin{tabular}{|c||c|c|}
    \hline
    \textbf{Word}  & $\Pr\sbrack{w \vert \text{action}}$ & $\Pr\sbrack{w \vert \text{comedy}}$ \\\hline\hline
    fast           &  $\frac{2}{3}$  &  $\frac{1}{2}$ \\\hline
    couple         &  $\frac{0}{3}$  &  $\frac{2}{2}$ \\\hline
    shoot          &  $\frac{3}{3}$  &  $\frac{0}{2}$ \\\hline
    fly            &  $\frac{1}{3}$  &  $\frac{1}{2}$ \\\hline
  \end{tabular}
\end{table}

Table~\ref{tab:P02:NormalWordProb} shows the probabilities normalized with add\-/1 smoothing.

\begin{table}[h]
  \centering
  \caption{Jurafsky Exercise~4.2 word probabilities with add\-/1 smoothing}\label{tab:P02:NormalWordProb}
  \begin{tabular}{|c||c|c|}
    \hline
    \textbf{Word}  & $\Pr\sbrack{w \vert \text{action}}$ & $\Pr\sbrack{w \vert \text{comedy}}$ \\\hline\hline
    fast           &  $\frac{3}{4}$  &  $\frac{2}{3}$ \\\hline
    couple         &  $\frac{1}{4}$  &  $\frac{3}{3}$ \\\hline
    shoot          &  $\frac{4}{4}$  &  $\frac{1}{3}$ \\\hline
    fly            &  $\frac{2}{4}$  &  $\frac{2}{3}$ \\\hline
  \end{tabular}
\end{table}

Similar to part~(\ref{P02:B}), the denominator ${\Pr\sbrack{\vec{w}}}$ is identical for both classes so this normalizing factor can be ignored.  Therefore:

\begin{aligncustom}
  \Pr\sbrack{\text{action} \vert \vec{w}} &\propto \Pr\sbrack{\text{action}} \prod_{w \in \vec{w}} \Pr\sbrack{w \vert \text{action}} \\
                            &= 0.6 * \left( \frac{3 * 1 *4 * 2}{4^4} \right) \\
                            &\approx \boxed{0.0563}
\end{aligncustom}

\noindent
and

\begin{aligncustom}
  \Pr\sbrack{\text{comedy} \vert \vec{w}} &\propto \Pr\sbrack{\text{comedy}} \prod_{w \in \vec{w}} \Pr\sbrack{w \vert \text{comedy}} \\
                            &= 0.4 * \left( \frac{2 * 3 *1 * 2}{3^4} \right) \\
                            &\approx \boxed{0.0593}\text{.}
\end{aligncustom}

\noindent
Given the above proportional probabilities, this movie review is more likely to be a \boxed{\text{comedy}}.

  \begin{problem}
  \textnormal{[Word2Vec with gensim and nltk]} Professor Heidi Kaufman from the Department of English would like to conduct research on the black rebellions in nineteenth-century print culture. Using ``rebellion'' and ``slave'' as the keywords, she would like to search for these words in a large collection of articles in the nineteenth century to understand the time and location distributions of such words in the documents.

  Prof.\ Kaufman discussed with Fred, a student in the Department of Computer and Information Science to see if he could help to do this task. Fred suggested that Prof.\ Kaufman could extend the keyword list to include similar words using the natural language processing technologies. This could help to improve the coverage of the search and provide a better estimations about the rebellions. In this problem, we will implement Fred’s idea using \texttt{word2vec}, the word representation/embedding model we learn in class.
\end{problem}

\begin{subproblem}
  Show the top ten words that are most similar to ``rebellion'' and ``slave'' based on the cosine similarity of the word vectors in the Word2Vec model you trained in step~2. Please include the cosine similarity for each word you report.
\end{subproblem}

For this problem, I used the default \texttt{word2vec} settings in \texttt{gensim}.  I also used the approach specified in the Kaggle link; \texttt{gensim} used all default settings.  Tables~\ref{tab:P03:Brown:Rebellion} and~\ref{tab:P03:Brown:Slave} show the 10~most similar words for rebellion and slave respectively.

\begin{table}[h]
  \centering
  \caption{Brown dataset top~10 most similar words for ``rebellion''}\label{tab:P03:Brown:Rebellion}
  \begin{tabular}{|c|c|}
    \hline
    \textbf{Word} & \textbf{Similarlity} \\\hline\hline
    incidence     & 0.9646 \\\hline
    Dominican     & 0.9646 \\\hline
    lacks         & 0.9639 \\\hline
    warrant       & 0.9632 \\\hline
    appestat      & 0.9626 \\\hline
    regime        & 0.9624 \\\hline
    vigorous      & 0.9613 \\\hline
    spectacular   & 0.9599 \\\hline
    underlying    & 0.9594 \\\hline
    X             & 0.9589 \\\hline
  \end{tabular}
\end{table}

\begin{table}[h]
  \centering
  \caption{Brown dataset top~10 most similar words for ``slave''}\label{tab:P03:Brown:Slave}
  \begin{tabular}{|c|c|}
    \hline
    \textbf{Word} & \textbf{Similarlity} \\\hline\hline
    finding       & 0.9639 \\\hline
    female        & 0.9602 \\\hline
    exciting      & 0.9564 \\\hline
    unusual       & 0.9562 \\\hline
    superiority   & 0.9552 \\\hline
    trap          & 0.9538 \\\hline
    released      & 0.9536 \\\hline
    displayed     & 0.9533 \\\hline
    substance     & 0.9529 \\\hline
    childhood     & 0.9525 \\\hline
  \end{tabular}
\end{table}

\begin{subproblem}
  Redo step~3, but instead of using the word2vec model trained in step 2, use the 300d pre-trained word2vec model provided by Google on \href{https://drive.google.com/file/d/0B7XkCwpI5KDYNlNUTTlSS21pQmM/edit}{Google Drive}. You will need to download this file (about 1.6GB) and load it (using \texttt{gensim}) to retrieve the most similar words. More information about this pre-trained model can be found at \url{https://code.google.com/archive/p/word2vec/} while the Kaggle link above might help to show how to load the vectors. Do you see any difference between the lists of words in steps~3 and~4?
\end{subproblem}

Tables~\ref{tab:P03:Word2Vec:Rebellion} and~\ref{tab:P03:Word2Vec:Slave} contain the 10~most similar words for rebellion and slave using Word2Vec.  Based on my own human understanding, the words identified by the pretrained \texttt{word2vec} seem to be much more closely related to both ``rebellion'' and ``slave'' than what was found using the Brown corpus.  This is most likely due to the pretrained \texttt{word2vec} corpus being substantially larger.

\begin{table}[h]
  \centering
  \caption{\texttt{word2vec} dataset top~10 most similar words for ``rebellion''}\label{tab:P03:Word2Vec:Rebellion}
  \begin{tabular}{|c|c|}
    \hline
    \textbf{Word} & \textbf{Similarlity} \\\hline\hline
    revolt        & 0.8357 \\\hline
    insurrection  & 0.7862 \\\hline
    rebellions    & 0.7244 \\\hline
    uprising      & 0.6939 \\\hline
    revolts       & 0.6581 \\\hline
    mutiny        & 0.6445 \\\hline
    uprisings     & 0.6043 \\\hline
    rebellious    & 0.5893 \\\hline
    rebelled      & 0.5875 \\\hline
    insurrections & 0.5765 \\\hline
  \end{tabular}
\end{table}

\begin{table}[h]
  \centering
  \caption{\texttt{word2vec} dataset top~10 most similar words for ``slave''}\label{tab:P03:Word2Vec:Slave}
  \begin{tabular}{|c|c|}
    \hline
    \textbf{Word} & \textbf{Similarlity} \\\hline\hline
    slaves            & 0.8382 \\\hline
    slavery           & 0.7260 \\\hline
    enslaved          & 0.7017 \\\hline
    enslaved\_Africans  & 0.6437 \\\hline
    slaver            & 0.6239 \\\hline
    slavers           & 0.5992 \\\hline
    Abraham\_Lincoln\_emancipation & 0.5779 \\\hline
    slave\_masters    & 0.5764 \\\hline
    abhorred\_slavery & 0.5687 \\\hline
    slaveholders      & 0.5585 \\\hline
  \end{tabular}
\end{table}

\end{document}
