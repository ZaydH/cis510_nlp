\begin{problem}
  \points{40} \textnormal{[Naive Bayes and smoothing]} Do exercises 4.1 and 4.2 in third (on-line) edition of the text (\url{https://web.stanford.edu/~jurafsky/slp3/ed3book.pdf}) (page 81). Show the intermediate steps in your calculation. Compute using probabilities, not logs of probabilities (so instead of adding logs of probabilities, you multiply probabilities).
\end{problem}

\begin{subproblem}\label{P02:A}
  Assume the following likelihoods for each word being part of a positive or negative movie review, and equal prior probabilities for each class.

  \begin{table}[h]
    \centering
    \caption{Jurafsky Exercise~4.1 word probabilities}\label{tab:}
    \begin{tabular}{ccc}
              & pos  & neg \\\hline
      I       & 0.09 & 0.16 \\
      always  & 0.07 & 0.06 \\
      like    & 0.29 & 0.06 \\
      foreign & 0.04 & 0.15 \\
      films   & 0.08 & 0.11 \\
    \end{tabular}
  \end{table}

  What  class  will  Naive  bayes  assign  to  the  sentence  ``I  always  like  foreign films.''
\end{subproblem}

For a class~${c \in \mathcal{C}}$, the Naive Bayes probability is:

\begin{aligncustom}
  \Pr \sbrack{c \vert \vec{w}} &= \frac{\Pr\sbrack{c} \Pr\sbrack{\vec{w} \vert c}}{\Pr\sbrack{\vec{w}}} \\
                               &= \frac{\Pr\sbrack{c} \prod_{w \in \vec{w}} \Pr\sbrack{w \vert c}}{\Pr\sbrack{\vec{w}}}\text{.}
\end{aligncustom}

\noindent
When doing the analysis in this problem, ${\Pr\sbrack{\vec{w}}}$~applies for all ${c \in \mathcal{C}}$ so it can be ignored in the calculations.  Similarly, since ${\Pr\sbrack{\text{pos}} = \Pr\sbrack{\text{neg}}}$, ${\Pr\sbrack{c}}$ can also be ignored.  Therefore:

\begin{aligncustom}
  \Pr\sbrack{\text{pos} \vert \vec{w}} &\propto 0.09 * 0.07 * 0.29 * 0.04 * 0.08 \\
                                       &\approx \boxed{5.85 * 10^{-6}}\text{.}
\end{aligncustom}

\begin{aligncustom}
  \Pr\sbrack{\text{neg} \vert \vec{w}} &\propto 0.16 * 0.06 * 0.06 * 0.15 * 0.11 \\
                                       &\approx \boxed{9.50 * 10^{-6}}\text{.}
\end{aligncustom}


\noindent
Therefore, it is more likely that the review ``I always like foreign films'' is \boxed{\text{neg}}.

\newpage
\begin{subproblem}\label{P02:B}
Given the following short movie reviews,  each labeled with a genre,  either comedy or action:

\begin{enumerate}
  \item fun, couple, love, love \hspace{6pt}\textbf{comedy}
  \item fast, furious, shoot \hspace{6pt}\textbf{action}
  \item couple, fly, fast, fun, fun \hspace{6pt}\textbf{comedy}
  \item furious, shoot, shoot, fun \hspace{6pt}\textbf{action}
  \item fly, fast, shoot, love \hspace{6pt}\textbf{action}
\end{enumerate}

\noindent
and a new document~D:

\noindent
\hspace{6pt}
fast, couple, shoot, fly

\noindent
compute the most likely class for~D. Assume a naive Bayes classifier and use add\-/1 smoothing for the likelihoods.
\end{subproblem}

From the above description, there are three \textbf{action} and two \textbf{comedy} movies.  Therefore,\linebreak ${\Pr\sbrack{\text{action}}=\frac{3}{5}=0.6}$ and ${\Pr\sbrack{\text{comedy}}=\frac{2}{5}=0.4}$.  The unnormalized word probabilities are shown in Table~\ref{tab:P02:UnnormalWordProb}.  The Bernoulli model for word probability was used where:

\begin{equation}\label{eq:P02:BernoulliWordProb}
  \Pr\sbrack{w_i \vert c} = \frac{\text{\# Docs Labeled } c \text{ containing } w_i}{\text{\# Docs Labeled } c}
\end{equation}

\noindent
Observe that only the words relevant to this problem (e.g.,~fast, couple, shoot, fly) are shown in the table.

\begin{table}[h]
  \centering
  \caption{Jurafsky Exercise~4.2 unnormalized word probabilities}\label{tab:P02:UnnormalWordProb}
  \begin{tabular}{|c||c|c|}
    \hline
    \textbf{Word}  & $\Pr\sbrack{w \vert \text{action}}$ & $\Pr\sbrack{w \vert \text{comedy}}$ \\\hline\hline
    fast           &  $\frac{2}{3}$  &  $\frac{1}{2}$ \\\hline
    couple         &  $\frac{0}{3}$  &  $\frac{2}{2}$ \\\hline
    shoot          &  $\frac{3}{3}$  &  $\frac{0}{2}$ \\\hline
    fly            &  $\frac{1}{3}$  &  $\frac{1}{2}$ \\\hline
  \end{tabular}
\end{table}

Table~\ref{tab:P02:NormalWordProb} shows the probabilities normalized with add\-/1 smoothing.  Since there are only two classes (e.g.,~\text{action} and \text{comedy}), ${k=2}$ is added to the denominator.

\begin{table}[h]
  \centering
  \caption{Jurafsky Exercise~4.2 word probabilities with add\-/1 smoothing}\label{tab:P02:NormalWordProb}
  \begin{tabular}{|c||c|c|}
    \hline
    \textbf{Word}  & $\Pr\sbrack{w \vert \text{action}}$ & $\Pr\sbrack{w \vert \text{comedy}}$ \\\hline\hline
    fast           &  $\frac{3}{5}$  &  $\frac{2}{4}$ \\\hline
    couple         &  $\frac{1}{5}$  &  $\frac{3}{4}$ \\\hline
    shoot          &  $\frac{4}{5}$  &  $\frac{1}{4}$ \\\hline
    fly            &  $\frac{2}{5}$  &  $\frac{2}{4}$ \\\hline
  \end{tabular}
\end{table}

\noindent
Similar to part~(\ref{P02:B}), the denominator ${\Pr\sbrack{\vec{w}}}$ is identical for both classes so this normalizing factor can be ignored.  Therefore:

\begin{aligncustom}
  \Pr\sbrack{\text{action} \vert \vec{w}} &\propto \Pr\sbrack{\text{action}} \prod_{w \in \vec{w}} \Pr\sbrack{w \vert \text{action}} \\
                            &= 0.6 * \left( \frac{3 * 1 *4 * 2}{5^4} \right) \\
                            &\approx \boxed{0.0230}
\end{aligncustom}

\noindent
and

\begin{aligncustom}
  \Pr\sbrack{\text{comedy} \vert \vec{w}} &\propto \Pr\sbrack{\text{comedy}} \prod_{w \in \vec{w}} \Pr\sbrack{w \vert \text{comedy}} \\
                            &= 0.4 * \left( \frac{2 * 3 *1 * 2}{4^4} \right) \\
                            &\approx \boxed{0.0188}\text{.}
\end{aligncustom}

\noindent
Given the above proportional probabilities, this movie review is more likely to be \boxed{\text{action}}.
