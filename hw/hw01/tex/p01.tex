\begin{problem}
  \points{40} \textnormal{[Document Similarity]} Suppose our pets have produced two documents:

  D1 = [\texttt{woof woof meow}] \\ D2 = [\texttt{woof woof squeak}].
\end{problem}

\begin{subproblem}\label{P01:A}
  \points{15} What is the cosine similarity of~D1 and~D2, not using idf weighting?
\end{subproblem}

  \begin{table}[h]
    \centering
    \caption{Problem~\ref{P01:A} Term Frequencies}\label{tab:P01:TermFreq}
    \begin{tabular}{|c|c|c|c|}
      \hline
      \textbf{Document} & \texttt{woof} & \texttt{meow} & \texttt{squeak} \\\hline\hline
      D1                & 2             & 1             & 0 \\\hline
      D2                & 2             & 0             & 1 \\\hline
    \end{tabular}
  \end{table}

The cosine similar is defined in Eq.~\eqref{eq:P01:Cosine}.

\begin{equation}\label{eq:P01:Cosine}
  \text{similarity} = \frac{D1 \cdot D2}{\norm{D1}_{2}\norm{D2}_{2}}
\end{equation}

\noindent
Therefore, given the term frequencies in Table~\ref{tab:P01:TermFreq}, the cosine similarity is:

\begin{aligncustom}
  \text{similarity} &= \frac{2 * 2 + 1 * 0 + 0 * 1}{\sqrt{2^2 + 1^2 + 0^2} * \sqrt{2^2 + 0^2 + 1^2}} \\
                    &= \frac{4}{5} = \boxed{0.8} \text{.}
\end{aligncustom}

\begin{subproblem}\label{P01:B}
  \points{15} What is the cosine similarity if idf weighting is used?
\end{subproblem}

The inverse document frequency,~$idf_i$, for the $i^{\text{th}}$ word is defined in Eq.~\eqref{eq:P01:IDF}, where $N$ is the number of document (e.g.,~size of the corpus) while $n_i$~is the number of documents containing the $i^{\text{th}}$~term.

\begin{equation}\label{eq:P01:IDF}
  idf = \log \left( \frac{N}{n_i} \right)
\end{equation}

\noindent
The weighting for each of the words in D1 and~D2 is shown in Table~\ref{tab:P01:IDF}.

  \begin{table}[h]
    \centering
    \caption{Problem~\ref{P01:B} Inverse Document Frequencies}\label{tab:P01:IDF}
    \begin{tabular}{|c|c|c|c|}
      \hline
          & \texttt{woof} & \texttt{meow}  & \texttt{squeak} \\\hline\hline
      IDF & 0             & $\log 2$       & $\log 2$ \\\hline
    \end{tabular}
  \end{table}

\noindent
The TF-IDF vectors are shown in Table~\ref{tab:P01:TFIDF}; the two vectors are orthogonal so their cosine similarity is~\boxed{0}.

  \begin{table}[h]
    \centering
    \caption{Problem~\ref{P01:A} Term Frequencies}\label{tab:P01:TFIDF}
    \begin{tabular}{|c|c|c|c|}
      \hline
      \textbf{Document} & \texttt{woof} & \texttt{meow} & \texttt{squeak} \\\hline\hline
      D1                & 0             & $\log 2$      & 0 \\\hline
      D2                & 0             & 0             & $\log 2$ \\\hline
    \end{tabular}
  \end{table}

\begin{subproblem}\label{P01:C}
  \points{10} How would the answer to~(\ref{P01:B}) change if we added a third document:

  D3 = [\texttt{meow squeak}]

  \noindent
  to the collection?
\end{subproblem}

D3~changes the IDF such that each word appears in two of the three documents.  Therefore, IDF can be treated as a constant scalar that cancels out when used in the similarity calculations.  This means the cosine similarity is the same as in part~(\ref{P01:A}) and equals~$\boxed{0.8}$.
