\documentclass{article}

\newcommand{\name}{Zayd Hammoudeh}
\newcommand{\course}{CIS510: Natural Lang. Processing}
\newcommand{\assnName}{HW\#2}
\newcommand{\dueDate}{November~4, 2019}

\usepackage[margin=1in]{geometry}
\usepackage[skip=4pt]{caption}      % ``skip'' sets the spacing between the figure and the caption.
\usepackage{tikz}
\usetikzlibrary{arrows.meta,decorations.markings,shadows,positioning,calc}
\usepackage{pgfplots}               % Needed for plotting
\pgfplotsset{compat=newest}
\usepgfplotslibrary{fillbetween}    % Allow for highlighting under a curve
\usepackage{amsmath}                % Allows for piecewise functions using the ``cases'' construct
\usepackage{siunitx}                % Allows for ``S'' alignment in table to align by decimal point

\usepackage[obeyspaces,spaces]{url} % Used for typesetting with the ``path'' command
\usepackage[hidelinks]{hyperref}    % Make the cross references clickable hyperlinks
\usepackage[bottom]{footmisc}       % Prevents the table going below the footnote
\usepackage{nccmath}                % Needed in the workaround for the ``aligncustom'' environment
\usepackage{amssymb}                % Used for black QED symbol
\usepackage{bm}                     % Allows for bolding math symbols.
\usepackage{tabto}                  % Allows to tab to certain point on a line
\usepackage{float}
\usepackage{subcaption}             % Allows use of the ``subfigure'' environment
\usepackage{enumerate}              % Allow enumeration other than just numbers

\usepackage[noend]{algpseudocode}
\usepackage[Algorithm,ruled]{algorithm}
\algnewcommand\algorithmicforeach{\textbf{for each}}
\algdef{S}[FOR]{ForEach}[1]{\algorithmicforeach\ #1\ \algorithmicdo}

%---------------------------------------------------%
%     Define Distances Used for Document Margins    %
%---------------------------------------------------%

\newcommand{\hangindentdistance}{1cm}
\newcommand{\defaultleftmargin}{0.25in}
\newcommand{\questionleftmargin}{-.5in}

\setlength{\parskip}{1em}
\setlength{\oddsidemargin}{\defaultleftmargin}

%---------------------------------------------------%
%      Configure the Document Header and Footer     %
%---------------------------------------------------%

% Set up page formatting
\usepackage{todonotes}
\usepackage{fancyhdr}                   % Used for every page footer and title.
\pagestyle{fancy}
\fancyhf{}                              % Clears both the header and footer
\renewcommand{\headrulewidth}{0pt}      % Eliminates line at the top of the page.
\fancyfoot[LO]{\course\ -- \assnName}   % Left
\fancyfoot[CO]{\thepage}                % Center
\fancyfoot[RO]{\name}                   % Right

%---------------------------------------------------%
%           Define the Title Page Entries           %
%---------------------------------------------------%

\title{\textbf{\course\ -- \assnName}}
\author{\name}

%---------------------------------------------------%
% Define the Environments for the Problem Inclusion %
%---------------------------------------------------%

\usepackage{scrextend}
\newcounter{problemCount}
\setcounter{problemCount}{0}  % Reset the subproblem counter

\newcounter{subProbCount}[problemCount]   % Reset subProbCount any time problemCount changes.
\renewcommand{\thesubProbCount}{\alph{subProbCount}}  % Make it so the counter is referenced as a number

\newenvironment{problemshell}{
  \begin{addmargin}[\questionleftmargin]{0em}
    \par%
    \medskip
    \leftskip=0pt\rightskip=0pt%
    \setlength{\parindent}{0pt}
    \bfseries
  }
  {
    \par\medskip
  \end{addmargin}
}
\newenvironment{problem}
{%
  \refstepcounter{problemCount} % Increment the subproblem counter.  This must be before the exercise to ensure proper numbering of claims and lemmas.
  \begin{problemshell}
    \noindent \textit{Exercise~\#\arabic{problemCount}} \\
  }
  {
  \end{problemshell}
  %  \setcounter{subProbCount}{0} % Reset the subproblem counter
}
\newenvironment{subproblem}
{%
  \begin{problemshell}
    \refstepcounter{subProbCount} % Increment the subproblem counter
    \setlength{\leftskip}{\hangindentdistance}
    % Print the subproblem count and offset to the left
    \hspace{-\hangindentdistance}(\alph{subProbCount}) \tabto{0pt}
  }
  {
  \end{problemshell}
}

% Change interline spacing.
\renewcommand{\baselinestretch}{1.1}
\newenvironment{aligncustom}
{ \csname align*\endcsname % Need to do this instead of \begin{align*} because of LaTeX bug.
  \centering
}
{
  \csname endalign*\endcsname
}

%---------------------------------------------------%
%       Define commands related to managing         %
%    floats (e.g., images) across multiple pages    %
%---------------------------------------------------%

\usepackage{placeins}     % Allows \FloatBarrier

% Prevent preceding floats going to this page
\newcommand{\floatnewpage}{\FloatBarrier\newpage}

% Add the specified input file and prevent any floated figures/tables going onto the same page as new input
\newcommand{\problemFile}[1]{
  \floatnewpage
  \input{#1}
}

../tex/global_macros.tex
\newcommand{\points}[1]{\textnormal{(#1~Points)}}


\begin{document}
  \maketitle

  \noindent
  \textbf{Name}: \name\\
  \textbf{Course}: \course\\
  \textbf{Assignment}: \assnName\\
  \textbf{Due Date}: \dueDate

  % \noindent
  % \textbf{Other Student Discussions}: I discussed the problems in this homework with the following student(s) below.  All write-ups were prepared independently.
  % \vspace{-1em}
  % \begin{itemize}
  %   \item Viet Lai
  %   \item Dave Patterson
  % \end{itemize}

  \section{Development Set Accuracy}

  The accuracy on the development set is:~$\boxed{94.485\%}$.

  \section{Implementation Notes}

  A primary goal was a neat, compact, efficient implementation.  As such, all operations use \texttt{numpy} \texttt{ndarray} objects and dot products.  This required significant discipline and pre-planning during implementation but yield clean non-verbose code.  Additionally, one requirement was to return a uniform distribution over unknown words.  To ensure compact code, I used the \texttt{defaultdict} class from the \texttt{collections} package to eliminate the need to have missing word checks.

  The combined train/dev data files were used to label the test set.  The labels file,~\texttt{wsj\_23.pos}, included with this submission used the combined file.

  A small dataset like the one provided with this homework is unlikely to have training examples that entail all valid transitions.  This makes add\-/1 smoothing particularly useful. However, I observed that an additive factor of~1 actually decreased accuracy by about~4\%.  As such, the additive scalar (referred to as~$\alpha$ in the slides) is~3E-3.

  I tried making the token case-insensitive, but it made the results worse.

  I also tried experimenting with non-uniform priors over unknown words.  I specifically tried using the true class priors (calculated from the training) set as the default vector for unknown words, but that made the performance slightly worse.  I settled on using transfer learning to predict unknown words.  Specifically I used the \texttt{nltk} to predict the POS of unknown words in isolation.  This improved development set accuracy by~\textasciitilde0.05\%.

  \section{Running the Program}

  The implementation was tested using Python~3.7.1.  The script is evoked on the command line as:

  \begin{center}
    \texttt{python viterbi.py train test out [--smooth]}
  \end{center}

  \noindent
  Each command line parameter is described below.

  \begin{itemize}
    \item \texttt{train}: Path to the \texttt{*.pos} training file
    \item \texttt{test}: Path to the (test) \texttt{*.words} file to be labeled
    \item \texttt{out}: Path for the file to write
    \item \texttt{--smooth}: Optionally enables ``add-1'' smoothing when calculating the likelihood and transition probabilities
  \end{itemize}
\end{document}
